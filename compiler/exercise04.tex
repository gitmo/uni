\documentclass[11pt,a4paper,ngerman]{article}
\usepackage[bottom=2.5cm,top=2.5cm]{geometry} 
\usepackage[german]{babel}
\usepackage[utf8]{inputenc}
\usepackage[T1]{fontenc} 
\usepackage{ae} 
\usepackage{amssymb} 
\usepackage{amsmath} 
\usepackage[pdftex]{graphicx}
\usepackage{fancyhdr}
\usepackage{fancyref}
\usepackage{xcolor}
\usepackage{paralist}
\usepackage{listings}
\usepackage[pdftex, bookmarks=false, pdfstartview={FitH}, linkbordercolor=white]{hyperref}
\usepackage{fancyhdr}
\pagestyle{fancy}
\fancyhead[R]{Übersetzerbau WS10/11: E. Fehr}
\fancyhead[L]{\today}
\fancyhead[C]{}
\fancyfoot{}
\fancyfoot[L]{}
\fancyfoot[C]{\thepage}
\renewcommand{\footrulewidth}{0.5pt}
\renewcommand{\headrulewidth}{0.5pt}
\setlength{\parindent}{0pt} 
\setlength{\headheight}{15pt}
%
\newcommand{\N}[0]{\mathbb{N}} 
\newcommand{\Z}[0]{\mathbb{Z}} 
\newcommand{\Q}[0]{\mathbb{Q}} 
\newcommand{\R}[0]{\mathbb{R}} 
\newcommand{\C}[0]{\mathbb{C}} 
\newcommand{\B}[0]{\mathbb{B}} 
\newcommand{\im}[0]{\mathit{i}}
\newcommand{\uber}[2]{{{#1} \choose {#2}}} 
\newcommand{\vect}[1]{\begin{pmatrix}#1\end{pmatrix}} 
%
\newcommand\zb{z.\,B.\ }
\renewcommand\dh{d.\,h.\ }
\newcommand\parbig{\par\bigskip}
\newcommand\parmed{\par\medskip}
\newcommand{\mailto}[1]{\href{mailto:#1}{#1}}
%
\title{Übungsblatt 04} 
%\author{von\\ XX,XX} 
\date{}
%
\begin{document} 
%
\lstset{language=Java, basicstyle=\ttfamily\fontsize{10pt}{10pt}\selectfont\upshape, commentstyle=\rmfamily\slshape, keywordstyle=\rmfamily\bfseries, breaklines=true, frame=single, xleftmargin=3mm, xrightmargin=3mm, tabsize=2}
%
\maketitle
\thispagestyle{fancy}


\section*{Aufgabe 1}
Von der Vorlesung,machen wir eine Erweiterung um Whitespaces zu überspringen, sodass nun auch Zeilen- und Blockkommentare ignoriert werden.
\parmed
\begin{lstlisting}
for ( ; ; peek = next input character) {
  if (peek is blank or tab) doNothing;
  else if (peek is newline) line = line + 1; 
  else if (peek is '/') {
    peek = next input character;
    
    /* a) Ignorieren von einzelne Zeilen comments */
    if (peek is '/') { 
      do {
      	peek = next input character;
      } while(peek is not newline);      
      continue;
    } 
    
    /* b) Ignorieren von mehrzeillen commentss*/
    if (peek is '*') {
      do {
      	peek = next input character;
      } while(peek is not '*' && peek+1 is not '/');
      peek = next input character;
      continue;  
  }
  break;
}
\end{lstlisting}

\section*{Aufgabe 2}
Erweiterung \texttt{scan()} Methodeum die Operatoren zu erkennen \texttt{<, >, <=, >=, ==, !=}.
\parmed
\begin{lstlisting}
Token scan() {
  // Ignore whitespaces and comments
  // Read digits
  // Read identifier and key words
  ...
  switch(peek) {
    case '<': case '>':
    	if (peek+1 is '=')
    	  return new Token(peek concat peek+1);
      return new Token(peek);
    case '=': case '!':
      if (peek+1 is '=')
        return new Token(peek concat peek+1);
      break;
    case default: // assume single character operator  
      return new Token(peek);
  }
}
\end{lstlisting}


\section*{Aufgabe 3}
\begin{itemize}
	\item[a)] $\left((\epsilon\mid a)b^*\right)^*$. Hier wird beschrieben (Sprache), die alle Wörter über dem Alphabet $\{a,b\}$ enthält, \dh die $a$s und $b$s können beliebig angeordnet sein.
	\item[b)] $(a\mid b)^* a(a\mid b)(a\mid b)$.  Hier wird beschrieben (Sprache) über dem Alphabet $\{a,b\}$, in der alle Wörter auf $aaa$, $aab$, $aba$ oder $abb$ enden. Also alle Wörter, deren drittletzter Buchstabe ein $a$ ist.
	\item[c)] $a^* ba^* ba^* ba^*$. Hier wird beschrieben (Sprache) über dem Alphabet $\{a,b\}$ deren Wörter genau drei $b$s enthalten.
\end{itemize}


\end{document}  