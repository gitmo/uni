\subsubsection*{Idee}

Idee war die Darstellung von \href{https://secure.wikimedia.org/wikipedia/de/wiki/Lissajous-Figuren}
{Lissajous-Figuren} auf dem Oszilloskop mit Hilfe des DMA-Controllers und  des Digital-Analog-Umwandlers.

Das Bild eines Oszilloskop kann mittels zwei Spannungseingängen manipuliert werden.
Punkte werden wie in einem Koordinaten-System mittels x- und y-Werten beschrieben.
Einer der beiden Eingänge des Oszilloskops steht für den x-Wert und der andere für den y-Wert.

Ziel war es nun über zwei Digital-Analog-Umwandler jeweils bereits berechnete Sinus-Werte in analoge Spannungen umzuwandeln und mit Hilfe dieser Spannungen Figuren auf dem Oszilloskop zu zeichnen. 

Hierzu wurden bei der Initialsierung des Mikrocontrollers die komplette Periode eines Sinus vorberechnet.
Da die Werte später über den Digital-Analog-Umwandler auf dem Oszilloskop ausgegeben werden,
wird der jeweilige Sinus-Wert noch auf einen Bereich von 0 bis 3V erhöht, sowie in eine 12 Bit Zahl konvertiert.

Der DMA-Controller wurde so konfiguriert, dass er zuvor initialisierte Timer als Trigger benutzt und den nächstfolgenden Sinus-Wert in das Register des Digital-Analog-Umwandlers kopiert. Dieser wandelt den Wert in eine analoge Spannung um.

Bei der Umsetzung wurde auch eine eigene Bibliothek für die jeweiligen Peripherieeinheiten des Mikrocontrollers geschrieben.

\subsubsection*{Quelltext}

\lstinputlisting[caption=project.h]{../MPP_WS1011/project/project.h}

\lstinputlisting[caption=project.c]{../MPP_WS1011/project/project.c}

\subsubsection*{Bibliotheken}

\lstinputlisting[caption=adu.h]{../MPP_WS1011/project/lib/adu.h}
\lstinputlisting[caption=dac.h]{../MPP_WS1011/project/lib/dac.h}
\lstinputlisting[caption=dma.h]{../MPP_WS1011/project/lib/dma.h}
\lstinputlisting[caption=mma.h]{../MPP_WS1011/project/lib/mma.h}
\lstinputlisting[caption=timer.h]{../MPP_WS1011/project/lib/timer.h}
\lstinputlisting[caption=uart.h]{../MPP_WS1011/project/lib/uart.h}
