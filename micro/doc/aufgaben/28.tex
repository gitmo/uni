\subsection
{\href{http://cst.mi.fu-berlin.de/intern/19606-P-MPP/Aufgaben/041102.html}
{Aufgabe 28: Sensordaten per Funk übertragen}}

Um Sensordaten per Funk übertragen zu können, müssen zwei Peripherieeinheiten  genutzt werden.
Zum einen der Transciever und zum anderen der Feuchtigkeits- / Temperatursensor.
In der Funktion \texttt{sendSensors()} werden die aktuellen Sensordaten ausgelesen und per Funk übertragen.
Um nun in jede Sekunde die Daten zu senden, wird genau jene Funktion in der \textit{Interrupt Service Routine} eines zuvor konfigurierten Timers aufgerufen.

\subsubsection*{Quelltext}

\lstinputlisting[linerange=8-999,firstnumber=8,caption=aufgabe28.c]
{../MPP_WS1011/aufgaben/aufgabe28.c}

%\subsubsection*{Beschleunigungssensor}

%\lstinputlisting[caption=mma.h]
%{../MPP_WS1011/aufgaben/mma.h}

%\lstinputlisting[linerange=9-999,firstnumber=9,caption=mma.c]
%{../MPP_WS1011/aufgaben/mma.c}

