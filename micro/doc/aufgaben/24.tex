\subsection
{\href{http://cst.mi.fu-berlin.de/intern/19606-P-MPP/Aufgaben/040802.html}
{Aufgabe 24: Der 3D Beschleunigungssensor}}

Hier wird die ADU genutzt, um die analogen Messwerte des 3D
Beschleunigungssensors einzulesen und weiterzuverarbeiten. Die
Komponenten im dreidimensionalen Raum ax, ay, az sind dabei auf den
Portleitungen 6.0, 6.1 und 6.2 geschaltet, die wir selektieren. Die
Leitungen 5.4 und 5.5 dienen zum Setzen des Messbereiches, wir wählen
1,5 G (Low, Low).

Nach dieser Initialisierung können die Beschleunigungs-Daten
aus der ADU ausgelesen und auf der seriellen Schnittstelle
ausgegeben. Vor jedem Auslesen wird gewartet bis die ADU das BUSY-Flag
gelöscht.

\subsubsection*{Quelltext}

\lstinputlisting[linerange=8-999,firstnumber=8,caption=aufgabe24.c]
{../MPP_WS1011/aufgaben/aufgabe24.c}
