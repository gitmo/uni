\subsection
{\href{http://cst.mi.fu-berlin.de/intern/19606-P-MPP/Aufgaben/040401.html}
{Aufgabe 12: Interrupt Quelle Taster}}
Jeder Tastendruck wird in derselben Interruptroutine abgehandelt. Je nachdem welche Taste und wie oft diese schon gedrückt wurde, wird die jeweilige LED an- bzw. ausgeschaltet. Die blinkende LED wird mittels einer Endlosschleife implementiert.
Um eine \textit{non-blocking} Lösung dafür zu erreichen, kann das \textit{togglen} der LED in eine Timer- oder Watchdog-Interruptroutine ausgelagert werden.
\lstinputlisting[linerange=9-49,firstnumber=9,caption=aufgabe12.c]
{../MPP_WS1011/aufgaben/aufgabe12.c}

\lstinputlisting[linerange=55-76,firstnumber=55,caption=interrupts\_a12.c]
{../MPP_WS1011/aufgaben/interrupts_a12.c}
