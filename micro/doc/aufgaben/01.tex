\subsection*
{\href{http://cst.mi.fu-berlin.de/intern/19606-P-MPP/Aufgaben/040101.html}
{Aufgabe 1: Portleitung als Output}}

Jeder Pin kann mittels den 8-Bit Registern PXDIR, PXOUT und PXIN konfiguiert beziehungsweise ausgelesen werden.
X ist dabei ein Wert zwischen 0 und 5 - stehend für den jeweiligen Port.

Mittels dem i-ten Bit des PXDIR-Registers wird der i-te Pin des X-Ports als Eingang oder Ausgang konfiguiert.
Je nachdem kann im PXIN-Register der Wert gelesen bzw. im PXOUT-Register geschrieben werden.

Die Pins selber können nur zwei Zustände annehmen: HIGH (logisch Eins) und LOW (logisch Null).
\subsubsection*{Erläuterung von Befehlen}
\lstinputlisting[linerange=29-41,firstnumber=29,caption=aufgabe01.c]
{../MPP_WS1011/aufgaben/aufgabe01.c}

\subsubsection*{Ampelschaltung}

\lstinputlisting[linerange=46-70,firstnumber=46,caption=aufgabe01.c]
{../MPP_WS1011/aufgaben/aufgabe01.c}
