\subsection
{\href{http://cst.mi.fu-berlin.de/intern/19606-P-MPP/Aufgaben/040703.html}
{Aufgabe 22: UART mit Empfangspuffer und Interrupt}}

Daten über die serielle Verbindung können auch mit Interrupts empfangen und gesendet werden.
Hierzu verwendet man meist ein Empfangsbuffer um einkommende Daten abspeichern zu können.
In unserem Fall verwenden wir ein einfaches \textit{Array}.
Für professionellere Anwendungszwecke wird hier meist ein \textit{Ringbuffer} eingesetzt.
Wenn wir nun mittels \texttt{\_bis\_SR\_register(GIE)} global die Interrupts einschalten,
wird für jedes einkommende Byte ein Interrupt generiert.
Im Interrupt selbst speichern wir das Byte im globalen Array ab,
und signalisieren unserem Hauptprogramm mittels der Variable \texttt{newLine},
ob wie einen Zeilenumbruch empfangen haben.
Das Hauptprogramm überprüft in jedem Schleifenschritt, ob wir einen String mit Zeilenumbruch empfangen haben
und sendet daraufhin die Länge des Strings zurück. 

\subsubsection*{Quelltext}

\lstinputlisting[caption=aufgabe22.h]
{../MPP_WS1011/aufgaben/aufgabe22.h}

\lstinputlisting[linerange=8-99,firstnumber=8,caption=aufgabe22.c]
{../MPP_WS1011/aufgaben/aufgabe22.c}

\lstinputlisting[linerange=57-80,firstnumber=57,caption=interrupts\_a22.c]
{../MPP_WS1011/aufgaben/interrupts_a22.c}
