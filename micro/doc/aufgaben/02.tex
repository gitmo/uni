\subsection*
{\href{http://cst.mi.fu-berlin.de/intern/19606-P-MPP/Aufgaben/040102.html}
{Aufgabe 2: Portleitung als Input}}

\subsubsection*{Erklärung der Befehlszeilen}
Aufgabe ist es einfache boolesche Ausdrücke auszuwerten.
\lstinputlisting[linerange=16-29,firstnumber=16,caption=aufgabe02.c]
{../MPP_WS1011/aufgaben/aufgabe02.c}

\subsubsection*{Tastenprogramm}
Je nachdem wie die Bit-Werte in der \texttt{P1IN}-Bit-Maske gesetzt sind, werden verschiedene LEDs an- bzw. ausgeschaltet. Mit anderen Worten je nachdem welche Taste gedrückt wird, leuchten andere LEDs.
Umgesetzt wird dies mit Hilfe einer \texttt{switch}-Anweisung.
\lstinputlisting[linerange=43-70,firstnumber=43,caption=aufgabe02.c]
{../MPP_WS1011/aufgaben/aufgabe02.c}
