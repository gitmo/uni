\subsection
{\href{http://cst.mi.fu-berlin.de/intern/19606-P-MPP/Aufgaben/041001.html}
{Aufgabe 26: Der Funktionsgenerator im DMA Betrieb}}

\subsubsection*{Vorgehensweise}

Bei dieser Aufgabenstellung, entfällt die Timer-ISR und stattdessen
werden die Sinuswerte mittels DMA an den Wandler geschrieben. Dennoch 
wird der Timer wie zuvor initialisiert, da er als Triggersignal dem DMA
signalisiert, wann der Speicherpointer auf den nächsten Werts
inkrementiert werden soll.

Zur Inititialisierung des DMA's legen wir im DMA-Control-Register die
Trigger-Quelle {\tt Timer\_B} fest. Das DMA-Modul 0 bekommt die
Startadresse und Größe des vorberechneten Sinus-Arrays
übergeben. Außerdem wird der DMA auf den Single-Transfer-Mode mit
inkrementeller Adressierung programmiert, dh. nach Übertragung der 100
Sample-Werte (und 100 Trigger-Events) ist Schluss. Als Zieladresse
soll direkt in das DA-Wandler-Datenregister geschrieben. Zum Schluss
wird der DMA eingeschaltet. Zusätzlich eine ISR bereitgestellt, die nach
Ende eines Single-Transfers-Durchlaufs den DMA mit dem Enable-Bit
erneut einschaltet.

\subsubsection*{Quelltext}
 
\lstinputlisting[linerange=8-999,firstnumber=8,caption=aufgabe26.c]
{../MPP_WS1011/aufgaben/aufgabe26.c}

%%% Local Variables: 
%%% mode: latex
%%% TeX-master: "../Main"
%%% End: 
