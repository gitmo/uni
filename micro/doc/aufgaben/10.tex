\subsection*
{\href{http://cst.mi.fu-berlin.de/intern/19606-P-MPP/Aufgaben/040302.html}
{Aufgabe 10: Nutzung des Watchdog bei lokalen Problemen}}

\subsubsection*{Programm}
Wird die äußere While-Schleife, in der der Watchdogzähler zurückgesetzt wird, zu lange
blockiert, ohne dass dies im blockierendem Code beachtet wird, kann es zu unerwarteten
Resets durch den Watchdog kommen. Abhilfe ist hier natürlich in der innere Schleife, die
während eines Tastendrucks blockt, den Counter gesondert zu löschen.


\subsubsection*{Quellcode}
\lstinputlisting[linerange=9-45,firstnumber=9,caption=aufgabe10.c]
{../MPP_WS1011/aufgaben/aufgabe10.c}

\begin{comment}
Wie können Sie registrieren und speichern, wann und an welcher Programmstelle der Watchdog das System neu gestartet hat.

Skizzieren Sie einen Lösungsansatz. Als Hilfestellung hier folgende Stichwörter:

    NMI-Interrupt
    Stackpointer
    Programcounter
    Softwarereset
    INFOMEM
\end{comment}
