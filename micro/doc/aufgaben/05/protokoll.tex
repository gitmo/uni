\subsection*{Aufgabe 5: Frequenzmessung der LFXT1CLK-, XT2CLK- und DCOCLK-Taktquelle}

\subsubsection*{Vorbereitung}
\begin{lstlisting}
void Aufgabe5() {
    P5SEL |= BIT4;    // P5 MCLK 
}\end{lstlisting}

\subsubsection*{Messwerte}
\paragraph{LFXT1CL} \texttt{(SELM = 11b)}
\begin{description}
	\item [LF-Mode] \texttt{(XTS = 0)}
		\begin{enumerate}
			\item minimal: 12,29kHz, DIVM=11b
			\item maximal: 32,769, DIVM=00
		\end{enumerate}

	\item [HF-Mode] \texttt{(XTS = 1)}
		\begin{enumerate}
			\item minimal: 636,5kHz, DIVM=11b
			\item maximal: 1,695 MHz, DIVM=00
		\end{enumerate}
\end{description}

\paragraph{XT2CL} \texttt{(SELM = 11b, XT2OFF=0)}
\begin{enumerate}
	\item minimal: 635 KHz, DIVM=11b
	\item maximal: 1,695 MHz, DIVM=00
\end{enumerate}

\paragraph{DCOCLK} \texttt{(SELM = 00b)}
\begin{enumerate}
	\item minimal: 393,2 kHz, DCO=000b, RSEL=000b, DIVM=11b, MOD = 00000b
	\item normal: 7,36 mHz, DCO=011b, RSEL=100b, DIVM=00b, MOD = 11101b
	\item maximal: 11,59Mhz, DCO=111b, RSEL=111b, DIVM=00b, MOD hat keinen Effekt 
\end{enumerate}

\paragraph{Erläutern Sie, wie der externe Widerstand für den DCOCLK-Taktgenerator nutzbar gemacht wir?}
Durch setzten des DCOR-Bits im BCSCTL2-Bit-Feld kann der externe Widerstand nutzbar gemacht werden.

\paragraph{Welchen Einfluss hat der Widerstand auf den DCOCLK-Taktgenerator?}
Der Widerstand verringert den Einfluss der Temperatur auf die Zuverlässigkeit des Taktgenerators. RC-Glied wieder niederohmiger und somit werden die Flanken deutlicher,da der Kondensator sich wieder schnell aufladen kann.

