\subsection
{\href{http://cst.mi.fu-berlin.de/intern/19606-P-MPP/Aufgaben/040301.html}
{Aufgabe 9: Nutzung des Watchdog}}

\subsubsection*{Vorbereitung}

Kapitel 10 des \emph{User Guide}s beschreibt Funktionsweise des
WDT-Moduls und die Register zu dessen Ansteuerung. Für diese Aufgabe
sind folgende Bits des Kontrollregisters WDTCTL relevant:

\begin{description}
    \item [WDTPW]    Passwortschutz (mehr in Aufgabe 11)
    \item [WDTHOLD]  Der User-Guide empfielt das WDT-Modul vor einem
Intervallwechsel anzuhalten
    \item [WDTCNTCL] Setzt den Timer-Zähler auf
0 zurück \item [WDTSSEL]  Auswahl des Taktquelle (0 SMCLK, 1 ACLK)
    \item [WDTISx]   2 Bits selektieren den Teiler für die Taktquelle (0 für
$/32768$) \end{description}

\subsubsection*{Programm}

Ohne Programmierung der Taktquelle ist der Watchdog-Counter nach dem
Einschalten auf ein $32 ms$ Intervall initialisiert. Die LED würde nie
eingeschaltet.

Die ACLK-Taktquelle des \emph{Basic Clock Moduls} wird auf den Vorteiler
$/8$ programmiert ($4096 Hz$) und für den Watchdog-Timer ausgewählt. Das
Watchdog-Interval wird nach $32768 / 4096 Hz = 8 s$ erreicht, d.h. die
LED wird bei 0,5 Sek. Schaltzyklus 8-mal blinken, falls der
Watchdog-Counter nicht zurückgesetzt wurde.

Es ist nun völlig ausreichend den Watchdog pro Schleifendurchlauf (Dauer
$1 s$) über \texttt{WDTCNTCL} zurückzusetzen.

\subsubsection*{Quelltext}

\lstinputlisting[linerange=9-40,firstnumber=9,caption=aufgabe09.c]
{../MPP_WS1011/aufgaben/aufgabe09.c}
