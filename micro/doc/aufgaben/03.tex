\subsection*{
\href{http://cst.mi.fu-berlin.de/intern/19606-P-MPP/Aufgaben/040103.html}
{Aufgabe 3: Ampelsteuerung}}
Eine Fußgängerampel kann in unterschiedliche Zustände unterteilt werden: ausgeschaltet, gelb, rot, rot-gelb und grün.
Um die jeweiligen Zustände lesbar zu repräsentieren wird ein \texttt{enum} verwendet.
In einer Endlosschleife wird dann einfach nur überprüft in welchem Zustand wir uns befinden und dieser dementsprechend angepasst.
\lstinputlisting[linerange=9-83,firstnumber=9,caption=aufgabe03.c]
{../MPP_WS1011/aufgaben/aufgabe03.c}

