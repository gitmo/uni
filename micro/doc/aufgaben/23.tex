\subsection
{\href{http://cst.mi.fu-berlin.de/intern/19606-P-MPP/Aufgaben/040801.html}
{Aufgabe 23: Messung einer Gleichspannung}}

Zunächst wird der Timer, wie in früheren Aufgaben, auf Sekundentakt
programmiert. Anschließend wird die ADU-Einheit initialisiert:
Taktquelle, Sample and Hold Modus mit 256 Zyklen, Portleitung 6.4
selektiert und auf Input geschaltet.

Alles weitere wird in der Timer-ISR einmal pro Sekunde ausgeführt:

\begin{itemize}
\item{} Ein neuer Sample-Lauf auf dem AD-Wandler gestartet.
\item{} Die ADU wird ausgelesen.
\item{} Der gemessene Spannungswert wird an die serielle Schnittstelle
  mit der bekannten Funktion {\tt uart1\-\_put\-\_str} gesendet.
\item{} Die LED's werden, je nach Spannungswert geschaltet.
\end{itemize}

Beim Auslesen der ADU über das Datenregister ADC12MEM0 muss der 12-Bit Wert auf den Spannungsbereich
von 0 ... 3 V skaliert werden, d.h.: $U = \frac{\text{ADC12MEM0}}
{(2^{12}-1)}  * 3 V$

\subsubsection*{Quelltext}

\lstinputlisting[linerange=8-99,firstnumber=8,caption=aufgabe23.c]
{../MPP_WS1011/aufgaben/aufgabe23.c}

%%% Local Variables: 
%%% mode: latex
%%% TeX-master: "../Main"
%%% End:

