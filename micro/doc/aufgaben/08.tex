\subsection
{\href{http://cst.mi.fu-berlin.de/intern/19606-P-MPP/Aufgaben/040204.html}
{Aufgabe 8: Abarbeitungszeit eines Befehls}}

Mit Hilfe eines digitalen Taktzählers wird die Zeit zwischen unterschiedlichen Spannungswerten am Port \texttt{P5OUT} gemessen.
\subsubsection*{Messungen}
\begin{itemize}
    \item bei steigendender Flanke: $0,7 \mu s$
    \item bei fallender Flanke: $0,9 \mu s$
\end{itemize}
Bei steigender Flanke wird die Dauer von zwei \texttt{XOR}-Instruktionen gemessen.
Währenddessen bei fallender Flanke alle Befehle vom zweiten \texttt{XOR} bis zum erneuten Schleifeneintritt und der ersten \texttt{XOR}-Instruktionen gemessen werden. Dies beinhaltet einen weiteren \texttt{JMP}-Befehl und dauert somit länger.
\subsubsection*{Quelltext}

\lstinputlisting[linerange=10-20,firstnumber=10,caption=aufgabe08.c]
{../MPP_WS1011/aufgaben/aufgabe08.c}
