%
%  Mikroprozessor Praktikum Protokolle
%
%  Sebastian Raitza, Nico von Geyso 2010
%
\documentclass[11pt,german]{scrartcl}

% See geometry.pdf to learn the layout options.
\usepackage{geometry}
\geometry{a4paper}

% To begin paragraphs with an empty line
\usepackage[parfill]{parskip}

% Use utf-8 encoding for foreign characters
\usepackage[utf8]{inputenc}
\usepackage[T1]{fontenc}

% Setup for fullpage use
\usepackage{fullpage}

% More symbols
\usepackage{amsmath}
\usepackage{amssymb}
%\usepackage{latexsym}

% Surround parts of graphics with box
\usepackage{boxedminipage}

% Package for including code in the document
\usepackage{listings}
\usepackage{color}
\usepackage{moreverb}
\usepackage{courier}


% This is now the recommended way for checking for PDFLaTeX:
\usepackage{ifpdf}

\ifpdf
\usepackage[pdftex]{graphicx}
\else
\usepackage{graphicx}
\fi

\ifpdf
\DeclareGraphicsExtensions{.pdf, .jpg, .tif}
\else
\DeclareGraphicsExtensions{.eps, .jpg}
\fi

\title{Mikroprozessor Praktikum\\Protokolle 1-15}
\author{Sebastian Raitza, Nico von Geyso}

% Activate to display a given date or no date
%\date{}

%%%%%%%%%%%%%%%%%%%%%%%%%%%%%%%%%%%%%%%%%%%%%%%%%%%%%%%%%%%%%%%%%%

\renewcommand{\labelenumi}{\alph{enumi})}
\renewcommand{\labelenumii}{$\bullet$}

% define a macro ausgabe which takes as argument
% your ausgabe.txt file’s name 
\newcommand{\ausgabe}[1] {\hrule\small\verbatiminput{#1}\normalsize\hrule}

\definecolor{light-gray}{gray}{0.95}

\lstset{
  language=C,               % choose the language of the code
  basicstyle=\ttfamily\scriptsize,   % size of fonts used for the code
  numbers=left,             % where to put the line-numbers
  numberstyle=\scriptsize,        % size of fonts for used line-numbers
  stepnumber=1,             % step between line-numbers
  numbersep=5pt,            % how far the line-numbers are from the code
  backgroundcolor=\color{white},  % background color, add \usepackage{color}
  showspaces=false,         % show spaces adding particular underscores
  showstringspaces=false,   % underline spaces within strings
  showtabs=false,           % show tabs within strings adding particular underscores
  frame=single,             % adds a frame around the code
  tabsize=4,                % sets default tabsize to 2 spaces
  captionpos=b,             % sets the caption-position to bottom
  breaklines=true,          % sets automatic line breaking
  breakatwhitespace=false,  % sets if automatic breaks should only happen at whitespace
  escapeinside={\%*}{*)},   % if you want to add a comment within your code
  frameround=tttt,
  extendedchars=true,
  literate=%
    {Ö}{{\"O}}1
    {Ä}{{\"A}}1
    {Ü}{{\"U}}1
    {ß}{{\ss}}2
    {ü}{{\"u}}1
    {ä}{{\"a}}1
    {ö}{{\"o}}1
    {°}{{}}0
}

\begin{document}

\maketitle

\section{Modulbeschreibung des Mikroprozessor-Praktikum}
Die überwältigende Mehrheit zukünftiger Computersysteme wird durch miteinander kommunizierende, eingebettete Systeme geprägt sein.
Diese finden sich in Maschinensteuerungen, Haushaltsgeräten, Kraftfahrzeugen, Flugzeugen, intelligenten Gebäuden etc. und
werden zukünftig immer mehr in Netze wie dem Internet eingebunden sein.
Das Praktikum wird auf die Architektur eingebetteter Systeme eingehen und die Unterschiede zu traditionellen PC-Architekturen 
(z.B. Echtzeitfähigkeit, Interaktion mit der Umgebung) anhand praktischer Beispiele aufzeigen. 
Das Praktikum basiert auf einem MSP430 Mikrocontrollersystem. 
Schwerpunkte des in einzelne Versuche gegliederten Praktikums sind: 
\begin{enumerate}
    \item Registerstrukturen
    \item Speicherorganisation
    \item hardwarenahe Assembler- und Hochsprachenprogrammierung
    \item I/O-System- und Timer-Programmierung
    \item Interrupt-System
    \item Watchdog-Logik
    \item Analogschnittstellen
    \item Bussystemanbindung von Komponenten
    \item Kommunikation (seriell, CAN-Bus, Ethernet, Funk und USB)
    \item Ansteuerung von Modellen und Nutzung unterschiedlichster Sensorik.
\end{enumerate}


\section{Entwicklungsboard MSB430H}
Auf dem zum entwickelten Board handelt es sich um das Entwicklungsboard MSB430H.
Der hier verwendete Mikrocontroller ist ein MSP430F1612 von Texas Instrument.
Dieser verfügt neben einem universellen Clock-System unter anderem über 55 kByte Flash, 5kByte Ram, USART und einem Watchdog.
Das Board ist zusätzlich mit einem 868MHz Transciever (CC1100),
sowie einem Luftfeuchtigkeits- und Temperatursensor (SHT11), einem 3D-Beschleunigungssensor (MMA7260Q)
und einer SD-Karte ausgestattet.

%%%%%%%%%%%%%%%%%%%%%%%%%%%%%%%%%%%%%%%%%%%%%%%%%%%%%%%%%%%%%%%%%%

\clearpage
\section*{I/O Portleitungen}
Ein Kennzeichen von Mikrocontrollern sind seine I/O-Portleitungen.
Auch bekannt unter dem Namen \textit{General Purpose Input/Output}.
Diese Pins können als Eingang um digitale Signale zu lesen fungieren
oder als Ausgang um andere Geräte zu kontrollieren oder mit ihnen zu kommunizieren.

Die Konfiguration dieser kann mittels Software programmiert werden
und überlässt so dem Entwickler große Freiheiten.
Bei manchen Mikrocontrollern können einige I/O-Ports desweiteren als Interruptquellen definiert werden.

Meist werden die I/O-Pins in unterschiedliche Ports gruppiert.
Ein Port hat meistens acht I/O-Pins.

Der von uns verwendete MSP430F1612 verfügt über sechs I/O-Ports mit jeweils acht Bit Breite.

\subsection*{Aufgabe 12: Interrupt Quelle Taster}

\lstinputlisting[linerange=9-49,firstnumber=9,caption=aufgabe12.c]
{../MPP_WS1011/aufgaben/aufgabe12.c}

\lstinputlisting[linerange=55-76,firstnumber=55,caption=interrupts\_a12.c]
{../MPP_WS1011/aufgaben/interrupts_a12.c}

\subsection*{Aufgabe 12: Interrupt Quelle Taster}

\lstinputlisting[linerange=9-49,firstnumber=9,caption=aufgabe12.c]
{../MPP_WS1011/aufgaben/aufgabe12.c}

\lstinputlisting[linerange=55-76,firstnumber=55,caption=interrupts\_a12.c]
{../MPP_WS1011/aufgaben/interrupts_a12.c}

\subsection*{Aufgabe 12: Interrupt Quelle Taster}

\lstinputlisting[linerange=9-49,firstnumber=9,caption=aufgabe12.c]
{../MPP_WS1011/aufgaben/aufgabe12.c}

\lstinputlisting[linerange=55-76,firstnumber=55,caption=interrupts\_a12.c]
{../MPP_WS1011/aufgaben/interrupts_a12.c}

\subsection*{Aufgabe 12: Interrupt Quelle Taster}

\lstinputlisting[linerange=9-49,firstnumber=9,caption=aufgabe12.c]
{../MPP_WS1011/aufgaben/aufgabe12.c}

\lstinputlisting[linerange=55-76,firstnumber=55,caption=interrupts\_a12.c]
{../MPP_WS1011/aufgaben/interrupts_a12.c}



%%%%%%%%%%%%%%%%%%%%%%%%%%%%%%%%%%%%%%%%%%%%%%%%%%%%%%%%%%%%%%%%%%

\clearpage
\section*{Clock System}
\subsection*{Aufgabe 12: Interrupt Quelle Taster}

\lstinputlisting[linerange=9-49,firstnumber=9,caption=aufgabe12.c]
{../MPP_WS1011/aufgaben/aufgabe12.c}

\lstinputlisting[linerange=55-76,firstnumber=55,caption=interrupts\_a12.c]
{../MPP_WS1011/aufgaben/interrupts_a12.c}

\subsection*{Aufgabe 12: Interrupt Quelle Taster}

\lstinputlisting[linerange=9-49,firstnumber=9,caption=aufgabe12.c]
{../MPP_WS1011/aufgaben/aufgabe12.c}

\lstinputlisting[linerange=55-76,firstnumber=55,caption=interrupts\_a12.c]
{../MPP_WS1011/aufgaben/interrupts_a12.c}

\subsection*{Aufgabe 12: Interrupt Quelle Taster}

\lstinputlisting[linerange=9-49,firstnumber=9,caption=aufgabe12.c]
{../MPP_WS1011/aufgaben/aufgabe12.c}

\lstinputlisting[linerange=55-76,firstnumber=55,caption=interrupts\_a12.c]
{../MPP_WS1011/aufgaben/interrupts_a12.c}

\subsection*{Aufgabe 12: Interrupt Quelle Taster}

\lstinputlisting[linerange=9-49,firstnumber=9,caption=aufgabe12.c]
{../MPP_WS1011/aufgaben/aufgabe12.c}

\lstinputlisting[linerange=55-76,firstnumber=55,caption=interrupts\_a12.c]
{../MPP_WS1011/aufgaben/interrupts_a12.c}


%%%%%%%%%%%%%%%%%%%%%%%%%%%%%%%%%%%%%%%%%%%%%%%%%%%%%%%%%%%%%%%%%%

\clearpage
\section*{Watchdog}
\subsection*{Aufgabe 12: Interrupt Quelle Taster}

\lstinputlisting[linerange=9-49,firstnumber=9,caption=aufgabe12.c]
{../MPP_WS1011/aufgaben/aufgabe12.c}

\lstinputlisting[linerange=55-76,firstnumber=55,caption=interrupts\_a12.c]
{../MPP_WS1011/aufgaben/interrupts_a12.c}

\subsection*{Aufgabe 12: Interrupt Quelle Taster}

\lstinputlisting[linerange=9-49,firstnumber=9,caption=aufgabe12.c]
{../MPP_WS1011/aufgaben/aufgabe12.c}

\lstinputlisting[linerange=55-76,firstnumber=55,caption=interrupts\_a12.c]
{../MPP_WS1011/aufgaben/interrupts_a12.c}

\subsection*{Aufgabe 12: Interrupt Quelle Taster}

\lstinputlisting[linerange=9-49,firstnumber=9,caption=aufgabe12.c]
{../MPP_WS1011/aufgaben/aufgabe12.c}

\lstinputlisting[linerange=55-76,firstnumber=55,caption=interrupts\_a12.c]
{../MPP_WS1011/aufgaben/interrupts_a12.c}


%%%%%%%%%%%%%%%%%%%%%%%%%%%%%%%%%%%%%%%%%%%%%%%%%%%%%%%%%%%%%%%%%%

\clearpage
\section*{Interrupt}
Mittels Interrupt kann aufgrund von gewissen Ereignissen die normale Programmabarbeitung unterbrochen werden.
Dies ist hilfreich um auf Ereignisse an I/O-Ports oder Peripheriegeräten sofort reagieren zu können.

Folgende Interruptquellen sind auf dem MSP430F1612 verfügbar:
\begin{enumerate}
    \item DA/Wandler und DMA
    \item I/O PORT 2
    \item USART1 Senden
    \item USART1 Empfangen
    \item I/O Port 
    \item TIMER A
    \item A/D Wandler   
    \item USART0 Senden
    \item USART1 Empfangen
    \item Watchdog Timer
    \item Comparator A
    \item TImer B
    \item NMI
    \item RESET
\end{enumerate}

Um mit Interrupts arbeiten zu können müssen diese global aktiviert sein.
Hierfür ist das GIE-Bit im SR-Register zuständig.
Desweiteren müssen in jeder Interruptquelle dieser nochmal erlaubt werden.

Wurde ein Interrupt ausgelöst so wird die dafür geschriebene Interrupt Service Routine (ISR) abgearbeitet. Anzumerken sei, dass die ISR dafür zuständig ist, dass jeweilige Interrupt-Flag wieder zurückzugesetzt wird.
Wird dies nicht getan, so springt der Microcontroller sofort wieder in die ISR und es kommt zu einer Endlosschleife.

\subsection*{Aufgabe 12: Interrupt Quelle Taster}

\lstinputlisting[linerange=9-49,firstnumber=9,caption=aufgabe12.c]
{../MPP_WS1011/aufgaben/aufgabe12.c}

\lstinputlisting[linerange=55-76,firstnumber=55,caption=interrupts\_a12.c]
{../MPP_WS1011/aufgaben/interrupts_a12.c}

\subsection*{Aufgabe 12: Interrupt Quelle Taster}

\lstinputlisting[linerange=9-49,firstnumber=9,caption=aufgabe12.c]
{../MPP_WS1011/aufgaben/aufgabe12.c}

\lstinputlisting[linerange=55-76,firstnumber=55,caption=interrupts\_a12.c]
{../MPP_WS1011/aufgaben/interrupts_a12.c}


%%%%%%%%%%%%%%%%%%%%%%%%%%%%%%%%%%%%%%%%%%%%%%%%%%%%%%%%%%%%%%%%%%

\clearpage
\section*{LPM-Modi}
Viele meist batteriebetriebene Mikrocontroller besitzen so genannte low power-Modi.
In diesen Modi ist der Takt stark reduziert oder die CPU gänzlich ausgeschaltet. Der Effekt davon ist ein minimaler Stromverbrauch.
Mittels Interrupts ist es möglich bei einer auschalteten CPU diese wieder zu aktiveren und die Programmabarbeitung fortzuführen.

Der MSP430F1612 besitzt sechs verschiedene Power-Modi:
\begin{description}
    \item [aktiver Modus]
    \item [LPM 4]  Nur eine Flanke an den Port1 und Port2 kann einen Interrupt auslösen und so den Mikrocontroller in den aktiven Modus versetzen.
    \item [LPM 3] Zzgl zu LPM4 kann ein Timer-Interrupt den Mikrocontroller erwecken. 
    \item [LPM 2]  
    \item [LPM 1] 
    \item [LPM 0] 
\end{description}

\subsection*{Aufgabe 12: Interrupt Quelle Taster}

\lstinputlisting[linerange=9-49,firstnumber=9,caption=aufgabe12.c]
{../MPP_WS1011/aufgaben/aufgabe12.c}

\lstinputlisting[linerange=55-76,firstnumber=55,caption=interrupts\_a12.c]
{../MPP_WS1011/aufgaben/interrupts_a12.c}

\subsection*{Aufgabe 12: Interrupt Quelle Taster}

\lstinputlisting[linerange=9-49,firstnumber=9,caption=aufgabe12.c]
{../MPP_WS1011/aufgaben/aufgabe12.c}

\lstinputlisting[linerange=55-76,firstnumber=55,caption=interrupts\_a12.c]
{../MPP_WS1011/aufgaben/interrupts_a12.c}

\end{document}
