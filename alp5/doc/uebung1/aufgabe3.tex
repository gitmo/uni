\section*{Aufgabe 3: Ein über TCP benutzbarer Dienst}

\subsection*{Chatroom.java}

Der Chatroom-Server wird mit einem optionalen Argument, dem Subject-String gestartet. Fehlt dieser bleibt Subject leer. Anschließend wird ein freier Port (oberhalb von 50000) gesucht und an ein Socket-Objekt gebunden.

\lstinputlisting[caption=Chatroom.java,firstnumber=27,linerange=27-34]{../../src/uebung1/aufgabe3/Chatroom.java}

Das Server besteht im wesentlichen aus einer großen Endlosschleife, in der nacheinander folgendes passiert:

\begin{itemize}
\item Warten (blockierend) bis sich ein neuer Client verbindet.
\item Ein neues Thread-Objekt {\tt workerClient} mit Logik zur Client-Kommunikation erzeugt.
\item Der Thread wird gestartet.
\end{itemize}

\subsection*{workerClient}

Zunächst wird der Client als ConcurrentEvent registriert und auf dem Socket auf einen Namensstring vom Client gewartet. Anschließend sendet der Server eine Willkommensnachricht an den Client inklusive Betreff.

\lstinputlisting[caption= workerClient in Chatroom.java,firstnumber=12,linerange=61-68]{../../src/uebung1/aufgabe3/Chatroom.java}

Des weiteren besteht {\tt workerClient} wiederum größtenteils aus einer großen Endlosschleife, die solange läuft bis der Client die Verbindung beendet hat.

\begin{itemize}
\item Blockieren bis der Client eine Nachricht gesendet hat
\item Überprüfung ob die Verbindung noch besteht (und die Nachricht gültig ist).
\item Client-Name und Nachricht wird an alle anderen verbundenen Clients propagiert.
\end{itemize}

\lstinputlisting[caption= workerClient in Chatroom.java,firstnumber=76,linerange=76-97]{../../src/uebung1/aufgabe3/Chatroom.java}

Nach Austritt der Schleife ist die Verbindung beendet und der Client wird mittels {\tt events.cancel(ticket)} aus dem Pool genommen und Lesepuffer und Socket geschlossen.

\subsection*{ChatClient.java}

Der Chat-Client wird mit drei Argument, dem Subject-String gestartet: Host, Port, und Name.

\begin{verbatim}
$ java –jar ChatClient.jar <localhost> <50000> <Klaus>
Socket initialized
Wecome Klaus! Subject is 'foo bar'
> 
\end{verbatim}

Der Client startet zwei Threads zum Senden bzw. Empfangen. Dazu werden vorher drei Puffer an die Systemeingabe, den Socket-Eingabe und die Socket-Ausgabe gebunden.

\lstinputlisting[caption=ChatClient.java,firstnumber=29,linerange=29-33]{../../src/uebung1/aufgabe3/ChatClient.java}

Der Sende-Thread liest von der Systemeingabe und schickt sie an den Server. 

\lstinputlisting[caption=inputWorker ChatClient.java,firstnumber=39,linerange=43-63]{../../src/uebung1/aufgabe3/ChatClient.java}

Der Empfangs-Thread liest analog über die Socket-Eingabe vom Server und gibt die Nachrichten auf der Konsole aus.

\lstinputlisting[caption=incomingWorker ChatClient.java,firstnumber=67,linerange=71-92]{../../src/uebung1/aufgabe3/ChatClient.java}
