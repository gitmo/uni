\subsubsection*{Aufgabe 1a}
Um alle Links aus einem HTML-Dokument zu extrahieren, muss zuerst einmal eine Verbindung zu diesem hergestellt werden.
Dies passiert mittels den Java-Klassen URL, sowie URLConnection. 
Hierbei wird gleichzeitig geprüft ob es sich bei dem Dokument auch wirklich um eine HTML-Datei handelt.
Ist dies der Fall so wird der komplette Inhalt ausgelesen und in einem String gespeichert.
\lstinputlisting[caption=Harvest.java,firstnumber=50,linerange=50-69]
{../../src/uebung6/aufgabe1/Harvest.java}

Nun wird der String Zeichen für Zeichen nach einem \textit{Anchor}-Element durchsucht.
Wurde eins gefunden so wird der Link mittels eines einfachen Regex-Ausdrucks ermittelt und zur Liste hinzugefügt.
\lstinputlisting[caption=Harvest.java,firstnumber=70,linerange=70-90]
{../../src/uebung6/aufgabe1/Harvest.java}
