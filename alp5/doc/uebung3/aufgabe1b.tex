\subsection*{b) Foreign RMI}

Die \textit{remote method invocation}-Technik ist nur in Kombination einer sogenannten \textit{rmiregistry} einsetzbar.
Diese verwaltet die instanzierten Objekte und ermöglicht so Fernaufrufe.
Da leider das Speichern von Instanzen in einer anderen \textit{rmiregistry} außer \textit{localhost}
nicht möglich ist, müssen wir mehrere Registries verwenden.
In unserem Falle sind dies die Registries auf dem jeweiligen Computer auf dem unsere Filter-Instanzen laufen,
sowie diejenige, auf der wir unser ForeignRMI-Klasse starten.

ForeignRMI hinterlegt in der lokalen Registry eine fernaufrufbare \textit{Set}-Datenstruktur.
\lstinputlisting[caption=ForeignRMI.java,firstnumber=36,linerange=36-45]
{../../src/uebung3/aufgabe1b/ForeignRMI.java}

Danach wird versucht sich über eigene Threads mit den Filter-Instanzen den Text zu analysieren.
Hierzu sind die jeweiligen Hostnamen auf dem eine Filter-Instanz läuft nötig.
\lstinputlisting[caption=ForeignRMI.java,firstnumber=51,linerange=51-60]
{../../src/uebung3/aufgabe1b/ForeignRMI.java}

Der Thread selber verbindet sich nun mit der \textit{rmiregistry} des jeweiligen Hosts und 
übergibt die zu filternden Daten, die globale Liste, sowie das Wörterbuch.
Mit Hilfe dieser Argumente können die Daten gefiltert und die Liste synchronisiert werden.
\lstinputlisting[caption=ForeignRMI.java,firstnumber=91,linerange=91-97]
{../../src/uebung3/aufgabe1b/ForeignRMI.java}
