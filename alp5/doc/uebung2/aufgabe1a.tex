\section*{Aufgabe 1}
\subsection*{a)}
\subsubsection*{Filter.java}
	Mit Hilfe der \texttt{StreamTokenizer}-Klasse wird der Text der zu analysierenden Datei in einzelne Wörter zerlegt und Schritt für Schritt abgearbeitet.
	Bei jedem Schritt wird das Wort über ein Wörterbuch abgeglichen. Enthält dieses das Wort, so wird es in der als Parameter übergebenen Namedpipe zwischengespeichert.
	\lstinputlisting[caption=Filter.java,firstnumber=27,linerange=27-42]
	{../../src/uebung2/aufgabe1/Filter.java}

\subsubsection*{Merge.java}
	Die \texttt{Merge}-Klasse bringt den Inhalt zweier Namedpipes zusammen und gibt danach die Häufigkeit der einzelnen Wörter textuell aus.
	Als Datenstruktur kommt eine \texttt{ConcurrentHashMap} zum Einsatz. Diese hat den Vorteil, dass keine Probleme aufgrund der Nebenläufigkeit enstehen können.
	\lstinputlisting[caption=Merge.java, firstnumber=16,linerange=16-16]
	{../../src/uebung2/aufgabe1/Merge.java}

	Nun lassen wir die Namedpipes nebenläufig über zwei Threads abarbeiten. Hierzu wird jede Zeile in die Datenstruktur eingetragen.
	\lstinputlisting[caption=Merge.java, firstnumber=30,linerange=30-48]
	{../../src/uebung2/aufgabe1/Merge.java}

	Nach erfolgreichem zusammenführen der Namedpipes wird letztendlich noch die \texttt{HashMap} selbst textuell ausgegeben.
	\lstinputlisting[caption=Merge.java, firstnumber=64,linerange=64-64]
	{../../src/uebung2/aufgabe1/Merge.java}




	\lstinputlisting[caption=Merge.java]
	{../../src/uebung2/aufgabe1/Merge.java}

