\subsection*{a) Filter und Merge mit \emph{Named Pipes}}
\subsubsection*{Filter.java}
	Mit Hilfe der \texttt{StreamTokenizer}-Klasse wird der Text der zu analysierenden Datei in einzelne Wörter zerlegt und Schritt für Schritt abgearbeitet.
	Bei jedem Schritt wird das Wort über ein Wörterbuch abgeglichen. Enthält dieses das Wort, so wird es in der als Parameter übergebenen \emph{Named Pipe} zwischengespeichert.
	\lstinputlisting[caption=Filter.java,firstnumber=27,linerange=27-42]
	{../../src/uebung2/aufgabe1/Filter.java}

\subsubsection*{Merge.java}
	Die \texttt{Merge}-Klasse bringt den Inhalt zweier \emph{Named Pipes} zusammen und gibt danach die Häufigkeit der einzelnen Wörter textuell aus.
	Als Datenstruktur kommt eine \texttt{ConcurrentHashMap} zum Einsatz. Diese hat den Vorteil, dass keine Probleme aufgrund der Nebenläufigkeit entstehen können.
	\lstinputlisting[caption=Merge.java, firstnumber=16,linerange=16-16]
	{../../src/uebung2/aufgabe1/Merge.java}

	Nun lassen wir die \emph{Named Pipes} nebenläufig über zwei Threads abarbeiten. Hierzu wird jede Zeile in die Datenstruktur eingetragen.
	\lstinputlisting[caption=Merge.java, firstnumber=30,linerange=30-48]
	{../../src/uebung2/aufgabe1/Merge.java}

	Nach erfolgreichem zusammenführen der \emph{Named Pipes} wird letztendlich noch die \texttt{HashMap} selbst textuell ausgegeben.
	\lstinputlisting[caption=Merge.java, firstnumber=67,linerange=67-73]
	{../../src/uebung2/aufgabe1/Merge.java}

{\sf TODO}

\lstinputlisting[caption= exercise1a.sh,language=bash,firstnumber=1,linerange=1-999]
 	{../../src/uebung2/aufgabe1/exercise1a.sh}


\subsection*{b) Merge auf entfernten Rechnern}

{\sf TODO}

\lstinputlisting[caption=remote.sh,language=bash,firstnumber=1,linerange=1-999]
	{../../src/uebung2/aufgabe1/remote.sh}

{\sf TODO}

\lstinputlisting[caption=exercise1b.sh,language=bash,firstnumber=1,linerange=1-999]
	{../../src/uebung2/aufgabe1/exercise1b.sh}

{\sf TODO}

\lstinputlisting[caption=Denglish.java,firstnumber=1,linerange=1-10]
	{../../src/uebung2/aufgabe1/Denglish.java}

{\sf TODO}

\begin{verbatim}
Output:
\end{verbatim}

	
\subsection*{c) Dienst-Architektur?}
{\sf TODO}

\begin{description}
\item[PRO:]
\item[CONTRA:]
\end{description}
