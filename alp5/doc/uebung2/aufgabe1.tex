\subsection*{a) Filter und Merge mit \emph{Named Pipes}}
\subsubsection*{Filter.java}
	Mit Hilfe der \texttt{StreamTokenizer}-Klasse wird der Text der zu analysierenden Datei in einzelne Wörter zerlegt und Schritt für Schritt abgearbeitet.
	Bei jedem Schritt wird das Wort über ein Wörterbuch abgeglichen. Enthält dieses das Wort, so wird es in der als Parameter übergebenen \emph{Named Pipe} zwischengespeichert.
	\lstinputlisting[caption=Filter.java,firstnumber=27,linerange=27-42]
	{../../src/uebung2/aufgabe1/Filter.java}

\subsubsection*{Merge.java}
	Die \texttt{Merge}-Klasse bringt den Inhalt zweier \emph{Named Pipes} zusammen und gibt danach die Häufigkeit der einzelnen Wörter textuell aus.
	Als Datenstruktur kommt eine \texttt{ConcurrentHashMap} zum Einsatz. Diese hat den Vorteil, dass keine Probleme aufgrund der Nebenläufigkeit entstehen können.
	\lstinputlisting[caption=Merge.java, firstnumber=16,linerange=16-16]
	{../../src/uebung2/aufgabe1/Merge.java}

	Nun lassen wir die \emph{Named Pipes} nebenläufig über zwei Threads abarbeiten. Hierzu wird jede Zeile in die Datenstruktur eingetragen.
	\lstinputlisting[caption=Merge.java, firstnumber=30,linerange=30-48]
	{../../src/uebung2/aufgabe1/Merge.java}

	Nach erfolgreichem zusammenführen der \emph{Named Pipes} wird letztendlich noch die \texttt{HashMap} selbst textuell ausgegeben.
	\lstinputlisting[caption=Merge.java, firstnumber=67,linerange=67-73]
	{../../src/uebung2/aufgabe1/Merge.java}

	Das folgende Bash-Skript automatisiert den Aufruf zweier Filter und das anschließende Mergen dieser Daten.
	Hierzu werden zwei Namedpipes verwendet. Als Parameter müssen die zwei zur filternden Dateien übergeben werden.
	\lstinputlisting[caption= exercise1a.sh,language=bash,firstnumber=1,linerange=1-999]
		{../../src/uebung2/aufgabe1/exercise1a.sh}


\subsection*{b) Merge auf entfernten Rechnern}

	Um die Lösung für Aufgabe 1b auszuführen muss folgendes Bash-Skript ausgeführt werden.
	Es braucht als Parameter zwei Hosts für die Abarbeitung, sowie zwei zu filternde Dateien.
	\lstinputlisting[caption=exercise1b.sh,language=bash,firstnumber=1,linerange=1-999]
	{../../src/uebung2/aufgabe1/exercise1b.sh}

	Da während der Bearbeitung der Übungszettel wir desöftern mit Pfad-Problemen zu kämpfen hatten,
	haben wir uns eine Remote-Bash-Datei geschrieben um das Filter-Programm letztendlich auf dem Remote-Server zu starten.
	Es wird also nicht direkt mittels \texttt{ssh -t host program} das Programm Filter aus dem Java-Quelltext aufgerufen
	sondern das Bash-Skript \texttt{remote.sh} übernimmt diese Aufgabe. 
	\lstinputlisting[caption=remote.sh,language=bash,firstnumber=1,linerange=1-999]
	{../../src/uebung2/aufgabe1/remote.sh}

	Als Datenstruktur zur Speicherung der gefilterten Wörter wurde für die Denglish-Klasse wieder eine ConcurrentHashMap gewählt.
	\lstinputlisting[caption=Denglish.java,firstnumber=18,linerange=18-18]
	{../../src/uebung2/aufgabe1/Denglish.java}

	Das Programm selbst arbeitet in zwei separaten Threads die zu filternden Dateien ab.
	Gestartet werden diese im Konstruktor der Klasse selbst.
	\lstinputlisting[caption=Denglish.java,firstnumber=36,linerange=36-38]
	{../../src/uebung2/aufgabe1/Denglish.java}

	Am Ende wird die Datenstruktur ConcurrentHashMap textuell ausgegeben:
	\lstinputlisting[caption=Denglish.java,firstnumber=102,linerange=102-105]
	{../../src/uebung2/aufgabe1/Denglish.java}


	Die Threads selber verbinden sich in der Methode \texttt{exec}  über \texttt{ssh} mit mit einem Remote Rechner und versuchen mit die Daten abzuarbeiten.
	Danach werden die Daten mit Hilfe der Methode \texttt{process} in die gemeinsame Datenstruktur gespeichert.
	\lstinputlisting[caption=Denglish.java,firstnumber=64,linerange=64-68]
	{../../src/uebung2/aufgabe1/Denglish.java}


%\begin{verbatim}
%Output:
%\end{verbatim}

	
\subsection*{c) Dienst-Architektur}
\begin{description}
\item[PRO:]
	Aufgrunddessen, dass Klienten keine wirkliche Antwort erhalten und der Broadcast eher einer \textit{event notification} gleicht könnte man argumentieren, dass es sich hierbei nicht um eine Dienst-Architektur handelt.

\item[CONTRA:]
	Gegen ein eventbasiertes System spricht, dass die Denglish-Klasse gleichzeitig ein Publisher als auch ein Subscriber ist.
	Er kennt also die Funktionalität des Subscribers. Noch dazu handelt es sich bei den Nachrichten nicht wirklich um sogenannte Events, da sie keinen Status enthalten.
\end{description}
