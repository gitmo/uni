\subsection*{rudimentärer Webserver}

Mit Hilfer der Java-Klasse \texttt{ServerSocket} wird ein Socket auf einem gegebenen Port eingerichtet.
Der Webserver lauscht nun in einer Endlosschleife auf eingehende Verbindungen und verarbeitet diese.
\lstinputlisting[caption=ServerWorker.java,firstnumber=118,linerange=118-131]
{../../src/uebung5/aufgabe1/ServerWorker.java}

Da es um HTTP-Anfragen handelt, wird der sogenannte HTTP-Header mit der Methode \textit{getHeaderFields()} ausgelesen und in einer \texttt{Map}-Datenstruktur gespeichert.
\lstinputlisting[caption=ServerWorker.java,firstnumber=34,linerange=34-73]
{../../src/uebung5/aufgabe1/ServerWorker.java}

Nun wird nun noch das HTML-Template mit dem Zähler über die Verbindung verschickt und der User-Agent gespeichert.
\lstinputlisting[caption=ServerWorker.java,firstnumber=136,linerange=136-163]
{../../src/uebung5/aufgabe1/ServerWorker.java}
Der User-Agent wird in einer Map-Datenstruktur gespeichert und letztendlich noch serialisiert und somit persistent gemacht.


