%
%  ALP V Übungszettel
%
\documentclass[11pt,german]{scrartcl}

% See geometry.pdf to learn the layout options.
\usepackage{geometry}
\geometry{a4paper}

% To begin paragraphs with an empty line
\usepackage[parfill]{parskip}

% Use utf-8 encoding for foreign characters
\usepackage[utf8]{inputenc}

% Setup for fullpage use
\usepackage{fullpage}

% Uncomment some of the following if you use the features
%
% Running Headers and footers
%\usepackage{fancyhdr}

% Multipart figures
%\usepackage{subfigure}

% More symbols
%\usepackage{amsmath}
%\usepackage{amssymb}
%\usepackage{latexsym}

% Surround parts of graphics with box
\usepackage{boxedminipage}

% Package for including code in the document
\usepackage{listings}
\usepackage{color}
\usepackage{moreverb}

% If you want to generate a toc for each chapter (use with book)
%\usepackage{minitoc}

% This is now the recommended way for checking for PDFLaTeX:
\usepackage{ifpdf}

%\newif\ifpdf
%\ifx\pdfoutput\undefined
%\pdffalse % we are not running PDFLaTeX
%\else
%\pdfoutput=1 % we are running PDFLaTeX
%\pdftrue
%\fi

\ifpdf
\usepackage[pdftex]{graphicx}
\else
\usepackage{graphicx}
\fi

\usepackage{courier}

\title{ALP V -- Übung 5}
\author{\bf Gruppe xxx, Tutor: xxx\\xxx, xxx}

% Activate to display a given date or no date
%\date{November 2, 2010}

\begin{document}

\ifpdf
\DeclareGraphicsExtensions{.pdf, .jpg, .tif}
\else
\DeclareGraphicsExtensions{.eps, .jpg}
\fi

\renewcommand{\labelenumi}{\alph{enumi})}
\renewcommand{\labelenumii}{$\bullet$}
\lstset{
	language=Java,                  % choose the language of the code
	                                % size of fonts used for the code
	basicstyle=\ttfamily\footnotesize,
	numbers=left,                   % where to put the line-numbers
	numberstyle=\footnotesize,      % size of fonts for used line-numbers
	stepnumber=1,                   % step between line-numbers
	numbersep=5pt,                  % how far the line-numbers are from the code
	backgroundcolor=\color{white},  % choose background color. You must add \usepackage{color}
	showspaces=false,               % show spaces adding particular underscores
	showstringspaces=false,         % underline spaces within strings
	showtabs=false,                 % show tabs within strings adding particular underscores
	frame=single,                   % adds a frame around the code
	tabsize=4,                      % sets default tabsize to 2 spaces
	captionpos=b,                   % sets the caption-position to bottom
	breaklines=true,                % sets automatic line breaking
	breakatwhitespace=false,        % sets if automatic breaks should only happen at whitespace
	escapeinside={\%*}{*)},         % if you want to add a comment within your code
  frameround=tttt,
  extendedchars=true,
  literate=%
    {Ö}{{\"O}}1
    {Ä}{{\"A}}1
    {Ü}{{\"U}}1
    {ß}{{\ss}}2
    {ü}{{\"u}}1
    {ä}{{\"a}}1
    {ö}{{\"o}}1
    {°}{{}}0
}


\maketitle

%%%%%%%%%%%%%%%%%%%%%%%%%%%%%%%%%%%%%%%%%%%%%%%%%%%%%%%%%%%%%%%%%%%%%%%%%%%%%%%%

\section*{Aufgabe 1 - Ein einfacher Web Server}
\subsubsection*{Aufgabe 1a}
Um alle Links aus einem HTML-Dokument zu extrahieren, muss zuerst einmal eine Verbindung zu diesem hergestellt werden.
Dies passiert mittels den Java-Klassen URL, sowie URLConnection. 
Hierbei wird gleichzeitig geprüft ob es sich bei dem Dokument auch wirklich um eine HTML-Datei handelt.
Ist dies der Fall so wird der komplette Inhalt ausgelesen und in einem String gespeichert.
\lstinputlisting[caption=Harvest.java,firstnumber=53,linerange=53-68]
{../../src/uebung6/aufgabe1/Harvest.java}

Nun wird der String Zeichen für Zeichen nach einem \textit{Anchor}-Element durchsucht.
Wurde eins gefunden so wird der Link mittels eines einfachen Regex-Ausdrucks ermittelt und zur Liste hinzugefügt.
\lstinputlisting[caption=JPanelStrategy.java,firstnumber=69,linerange=69-91]
{../../src/uebung6/aufgabe1/Harvest.java}


\subsection*{b) Foreign RMI}

\lstinputlisting[caption=ForeignRMI.java,firstnumber=1,linerange=1-999]
{../../src/uebung3/aufgabe1b/ForeignRMI.java}


\subsection*{Histogramm}
Die Klasse \textit{StatisticWorker} arbeitet ähnlich wie die Klasse \textit{ServerWorker}.
Auch hier werden mittels eines \textit{ServerSockets} eingehende Verbindungen abgearbeitet.
Im Unterschied wird hier aber nicht der \textit{User-Agent} zu Statistikzwecken gespeichert,
sondern die gesammelten Daten ausgegeben.
Hierzu wird die serialisierte Map-Datenstruktur geladen und formatiert ausgegeben.
\lstinputlisting[caption=StatisticWorker.java,firstnumber=75,linerange=75-98]
{../../src/uebung5/aufgabe1/StatisticWorker.java}


\section*{Aufgabe 2 - Ein spezialisierter Web Browser}
\subsection*{... und das Ganze nur mit Sockets?}

% Offensichtlich ist der Einsatz von RMI für die Lösung des obigen
% Textanalyse-Problems nicht gerade ein Kinderspiel.  Könnte man dann
% nicht auch gleich darauf verzichten und direkt mit Sockets arbeiten?  
% Beschreiben Sie für einen Programmierer, der sich mit Sockets auskennt,
% wie in diesem Fall vorzugehen wäre!

TODO

\end{document}
