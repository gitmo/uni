%
%  ALP V Übungszettel
%
\documentclass[11pt,german]{scrartcl}

% See geometry.pdf to learn the layout options.
\usepackage{geometry}
\geometry{a4paper}

% To begin paragraphs with an empty line
\usepackage[parfill]{parskip}

% Use utf-8 encoding for foreign characters
\usepackage[utf8]{inputenc}

% Setup for fullpage use
\usepackage{fullpage}

% Uncomment some of the following if you use the features
%
% Running Headers and footers
%\usepackage{fancyhdr}

% Multipart figures
%\usepackage{subfigure}

% More symbols
%\usepackage{amsmath}
%\usepackage{amssymb}
%\usepackage{latexsym}

% Surround parts of graphics with box
\usepackage{boxedminipage}

% Package for including code in the document
\usepackage{listings}
\usepackage{color}
\usepackage{moreverb}

% If you want to generate a toc for each chapter (use with book)
%\usepackage{minitoc}

% This is now the recommended way for checking for PDFLaTeX:
\usepackage{ifpdf}

%\newif\ifpdf
%\ifx\pdfoutput\undefined
%\pdffalse % we are not running PDFLaTeX
%\else
%\pdfoutput=1 % we are running PDFLaTeX
%\pdftrue
%\fi

\ifpdf
\usepackage[pdftex]{graphicx}
\else
\usepackage{graphicx}
\fi

\usepackage{courier}

\title{ALP V -- Übung 5}
\author{\bf Gruppe xxx, Tutor: xxx\\xxx, xxx}

% Activate to display a given date or no date
%\date{November 2, 2010}

\begin{document}

\ifpdf
\DeclareGraphicsExtensions{.pdf, .jpg, .tif}
\else
\DeclareGraphicsExtensions{.eps, .jpg}
\fi

\renewcommand{\labelenumi}{\alph{enumi})}
\renewcommand{\labelenumii}{$\bullet$}
\lstset{
	language=Java,                  % choose the language of the code
	                                % size of fonts used for the code
	basicstyle=\ttfamily\footnotesize,
	numbers=left,                   % where to put the line-numbers
	numberstyle=\footnotesize,      % size of fonts for used line-numbers
	stepnumber=1,                   % step between line-numbers
	numbersep=5pt,                  % how far the line-numbers are from the code
	backgroundcolor=\color{white},  % choose background color. You must add \usepackage{color}
	showspaces=false,               % show spaces adding particular underscores
	showstringspaces=false,         % underline spaces within strings
	showtabs=false,                 % show tabs within strings adding particular underscores
	frame=single,                   % adds a frame around the code
	tabsize=4,                      % sets default tabsize to 2 spaces
	captionpos=b,                   % sets the caption-position to bottom
	breaklines=true,                % sets automatic line breaking
	breakatwhitespace=false,        % sets if automatic breaks should only happen at whitespace
	escapeinside={\%*}{*)},         % if you want to add a comment within your code
  frameround=tttt,
  extendedchars=true,
  literate=%
    {Ö}{{\"O}}1
    {Ä}{{\"A}}1
    {Ü}{{\"U}}1
    {ß}{{\ss}}2
    {ü}{{\"u}}1
    {ä}{{\"a}}1
    {ö}{{\"o}}1
    {°}{{}}0
}


\maketitle

%%%%%%%%%%%%%%%%%%%%%%%%%%%%%%%%%%%%%%%%%%%%%%%%%%%%%%%%%%%%%%%%%%%%%%%%%%%%%%%%

\section*{Aufgabe 1 - Ein einfacher Web Server}
\section*{Aufgabe 1}
\subsection*{a)}
\subsubsection*{Filter.java}
	Mit Hilfe der \texttt{StreamTokenizer}-Klasse wird der Text der zu analysierenden Datei in einzelne Wörter zerlegt und Schritt für Schritt abgearbeitet.
	Bei jedem Schritt wird das Wort über ein Wörterbuch abgeglichen. Enthält dieses das Wort, so wird es in der als Parameter übergebenen Namedpipe zwischengespeichert.
	\lstinputlisting[caption=Filter.java,firstnumber=27,linerange=27-42]
	{../../src/uebung2/aufgabe1/Filter.java}

\subsubsection*{Merge.java}
	Die \texttt{Merge}-Klasse bringt den Inhalt zweier Namedpipes zusammen und gibt danach die Häufigkeit der einzelnen Wörter textuell aus.
	Als Datenstruktur kommt eine \texttt{ConcurrentHashMap} zum Einsatz. Diese hat den Vorteil, dass keine Probleme aufgrund der Nebenläufigkeit enstehen können.
	\lstinputlisting[caption=Merge.java, firstnumber=16,linerange=16-16]
	{../../src/uebung2/aufgabe1/Merge.java}

	Nun lassen wir die Namedpipes nebenläufig über zwei Threads abarbeiten. Hierzu wird jede Zeile in die Datenstruktur eingetragen.
	\lstinputlisting[caption=Merge.java, firstnumber=30,linerange=30-48]
	{../../src/uebung2/aufgabe1/Merge.java}

	Nach erfolgreichem zusammenführen der Namedpipes wird letztendlich noch die \texttt{HashMap} selbst textuell ausgegeben.
	\lstinputlisting[caption=Merge.java, firstnumber=64,linerange=64-64]
	{../../src/uebung2/aufgabe1/Merge.java}


\subsection*{b) Foreign RMI}

Die \textit{remote method invocation}-Technik ist nur in Kombination mit einer sogenannten \textit{rmiregistry} einsetzbar.
Diese verwaltet die instanzierten Objekte und ermöglicht so Fernaufrufe.
Da leider das Speichern von Instanzen in einer anderen \textit{rmiregistry} außer der Lokalen 
nicht möglich ist, müssen wir mehrere Registries verwenden.
In unserem Falle sind dies die Registries auf dem jeweiligen Computer auf dem unsere Filter-Instanzen laufen,
sowie diejenige, auf der wir unser ForeignRMI-Klasse starten.

ForeignRMI hinterlegt in der lokalen Registry eine fernaufrufbare \textit{Set}-Datenstruktur.
\lstinputlisting[caption=ForeignRMI.java,firstnumber=36,linerange=36-45]
{../../src/uebung3/aufgabe1b/ForeignRMI.java}

Danach wird versucht sich über eigene Threads mit den Filter-Instanzen den Text zu analysieren.
Hierzu sind die jeweiligen Hostnamen auf dem eine Filter-Instanz läuft nötig.
\lstinputlisting[caption=ForeignRMI.java,firstnumber=51,linerange=51-60]
{../../src/uebung3/aufgabe1b/ForeignRMI.java}

Der Thread selber verbindet sich nun mit der \textit{rmiregistry} des jeweiligen Hosts und 
übergibt die zu filternden Daten, die globale Liste, sowie das Wörterbuch.
Mit Hilfe dieser Argumente können die Daten gefiltert und die Liste synchronisiert werden.
\lstinputlisting[caption=ForeignRMI.java,firstnumber=91,linerange=91-97]
{../../src/uebung3/aufgabe1b/ForeignRMI.java}

Die Filter-Klasse ist relativ analog zur Aufgabe 1a, aber um Fernaufrufe erweitert.


\subsection*{Histogramm}
Die Klasse \textit{StatisticWorker} arbeitet ähnlich wie die Klasse \textit{ServerWorker}.
Auch hier werden mittels eines \textit{ServerSockets} eingehende Verbindungen abgearbeitet.
Im Unterschied wird hier aber nicht der \textit{User-Agent} zu Statistikzwecken gespeichert,
sondern die gesammelten Daten ausgegeben.
Hierzu wird die serialisierte Map-Datenstruktur geladen und formatiert ausgegeben.
\lstinputlisting[caption=StatisticWorker.java,firstnumber=75,linerange=75-98]
{../../src/uebung5/aufgabe1/StatisticWorker.java}


\section*{Aufgabe 2 - Ein spezialisierter Web Browser}
\subsection*{... und das Ganze nur mit Sockets?}

% Offensichtlich ist der Einsatz von RMI für die Lösung des obigen
% Textanalyse-Problems nicht gerade ein Kinderspiel.  Könnte man dann
% nicht auch gleich darauf verzichten und direkt mit Sockets arbeiten?  
% Beschreiben Sie für einen Programmierer, der sich mit Sockets auskennt,
% wie in diesem Fall vorzugehen wäre!

TODO

\end{document}
