\section{Architektur}

\begin{frame}{Architektur}
    \begin{center}
    \textbf{Überblick}
    \end{center}

    \begin{itemize}
        \item Von-Neumann-Architektur
        \item ursprünglich 16 Bit Word-Breite
        \item basiert auf der Instruktionsmenge des 8086 Mikrocontroller
        \item Abwärtskompatibilität mit Hilfe von verschiedenen Operationsmodi
    \end{itemize}
\end{frame}


\begin{frame}{Architektur}
    \begin{center}
    \textbf{Register}
    \end{center}

    \begin{itemize}
        \item Allzweckregister: \texttt{EAX, EBX, ECX, EDX} (früher: AX, BX, ...)
        \item Segmentregister: u.a. \texttt{CS}, \texttt{DS} und \texttt{SS}
        \item Zeige- und Indexregister: u.a. \texttt{ESP} und \texttt{EIP}
        \item Statusregister
    \end{itemize}
\end{frame}


\begin{frame}{Architektur}
    \begin{center}
    \textbf{Statusregister}
    \end{center}

    oft genutzte \textit{Flag}-Register
    \begin{itemize}
        \item overflow
        \item sign
        \item zero
        \item ...
    \end{itemize}

    \makebox{}

    Statusregister im Überblick (16 Bit)
    \begin{center}
        {\small 15}
        \begin{tabular}{|c|c|c|c|c|c|c|c|c|c|c|c|c|c|c|c|c|}
        \hline & & & & O & D & I & T & S & Z & & A & & P & & C \\
        \hline
        \end{tabular}
        {\small 0}
    \end{center}
\end{frame}

\begin{frame}{Architektur}
    \begin{center}
    \textbf{Arten der Adressierung}
    \end{center}

    \begin{itemize}
        \item Direktwertadressierung (Immediates): {\tt mov bx,FFF0h}
        \item Registeradressierung: {\tt mov    ax,bx}
        \item Direkte Adressierung: {\tt mov    ax,ds:index}
        \item Indirekte Registeradressierung: {\tt mov ax,[bx]}
        \item Indizierte Registeradressierung: {\tt mov [Array+bx],bx}
    \end{itemize}
\end{frame}
