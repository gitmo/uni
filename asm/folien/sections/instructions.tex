\section{Instruktionen}

\begin{frame}[fragile]{Daten kopieren}

\begin{center}
\textbf{\texttt{MOV}-Mnemonic}
\end{center}

Syntax
\begin{lstlisting}
mov    Ziel, Quelle
\end{lstlisting}

\makebox{}

Semantik
\begin{itemize}
    \item Daten in oder aus einem Register/Speicher kopieren
    \item einem Register einen Wert zu weisen
\end{itemize}

\makebox{}

Beispiel
\begin{lstlisting}
mov    eax, 5    ; Register eax Wert 5 zu weisen
mov    eax, ebx  ; Wert von Register ebx in eax kopieren
\end{lstlisting}

\end{frame}

\begin{frame}[fragile]{Mathematische und logische Operationen}

\begin{center}
\textbf{Mathematische Operationen}
\end{center}

verfügbare Operationen / Mnemonics
\begin{description}
    \item [INC] Inkrementierung
    \item [DEC] Dekrementierung
    \item [ADD] Addition
    \item [SUB] Subtraktion
    \item [...]
\end{description}

\makebox{}

Syntax
\begin{lstlisting}
Mnemonic    Ziel[, Quelle]
\end{lstlisting}

\makebox{}

Semantik
\begin{itemize}
    \item Ergebnis wird im Ziel-Register gespeichert

    \item auftretende Sonderfälle werden in den \textit{Flag}-Registern gespeichert
    \begin{itemize}
        \item Overflow
        \item Zero
        \item ...
    \end{itemize}
\end{itemize}
\end{frame}

\begin{frame}[fragile]{Mathematische und logische Operationen}

\begin{center}
\textbf{Beispiele}
\end{center}

Addition
\begin{lstlisting}
mov    eax, 5  ; Zuweisung eax = 5
add    eax, 5  ; 5+5 = 10
\end{lstlisting}

\makebox{}

Dekrementierung
\begin{lstlisting}
mov    eax, 1   ; Zuweisung eax = 1
dec    eax      ; 1 - 1 = 0
                ; Zero-Flag gesetzt
\end{lstlisting}

\end{frame}

\begin{frame}{Vergleiche}

\begin{center}
\textbf{Vergleiche}
\end{center}

Ergebnisse von Vergleichen werden in den \textit{Flag}-Registern gespeichert

\makebox{}

Vergleichs-Operationen
\begin{itemize}
    \item \textbf{binäre Subtraktion}\\
        \texttt{cmp arg1, arg2} $\leftrightarrow \text{arg1} - \text{arg2}$

    \item \textbf{logische VerUNDung}\\
        \texttt{test arg1, arg2} $\leftrightarrow \text{arg1} \wedge \text{arg2}$
\end{itemize}
\end{frame}

\begin{frame}[fragile]{Vergleiche}

Beispiel
\begin{lstlisting}
mov    eax, 5  ; eax = 5
cmp    eax, 5  ; Zero Flag gesetzt -> 5 == 5
\end{lstlisting}

\makebox{}

\textit{Flag}-Kombination und ihre Aussage über die Relation:

\makebox{}

\begin{small}
\begin{tabular}{|c|c|c|c|c|c|c|c|}
\hline
1. Operand & 2. Operand & Ergebnis &
CF\footnote{Carry Flag} &
OF\footnote{Overflow Flag} &
SF\footnote{Sign Flag} &
ZF\footnote{Zero Flag}
& Relation \\ \hline
5          & 5          & 0        & 0  & 0  & 0  & 1  & $\text{op1} = \text{op2}$ \\ \hline
10         & 5          & 5        & 0  & 0  & 0  & 0  & $\text{op1} > \text{op2}$ \\ \hline
5          & 10         & -5       & 0  & 0  & 1  & 0  & $\text{op1} < \text{op2}$ \\ \hline
\multicolumn{8}{|c|}{...} \\ \hline
\end{tabular}
\end{small}

\end{frame}

\begin{frame}[fragile]{Bedingte und unbedingte Sprünge}

\begin{center}
\textbf{Unbedingte Sprünge / \texttt{JMP}-Mnemonic}
\end{center}

\begin{itemize}
    \item ermöglicht das Anspringen von Adressen ferner Instruktionen
    \item Umgesetzt mittels Manipulation des \textit{Instruction Pointers}
\end{itemize}

\makebox{}

Beispiel
\begin{lstlisting}
jmp    exit   ; Springe zum Label exit
.
.
.
exit: ...     ; Label exit
\end{lstlisting}
\end{frame}

\begin{frame}[fragile]{Bedingte und unbedingte Sprünge}

\begin{center}
\textbf{Bedingte Sprünge}
\end{center}

\begin{itemize}
    \item Realisierung von aus Hochsprachen bekannten Konstrukten
    \begin{itemize}
        \item if/then, else
        \item switch
        \item while, for
        \item ...
    \end{itemize}
\end{itemize}

\makebox{}

Semantik
\begin{itemize}
    \item Sprung wird genommen, wenn die Bedingung erfüllt ist
    \item Bedingungsergebnis wird mit Hilfe der \textit{Flag}-Register ermittelt
\end{itemize}
\end{frame}


\begin{frame}
Oft benutzte bedingte Sprunganweisungen
\begin{description}
    \item [JZ] jump if zero
    \item [JG] jump if greater
    \item [JE] jump if equal
    \item [JL] jump if less
    \item [...]
\end{description}
\end{frame}


\begin{frame}[fragile]{bedingte Sprünge}

\begin{center}
\textbf{Beispiel}
\end{center}

Pseudocode
\begin{lstlisting}
if(ecx != edx)
{
    ...
}
//ifEnd
\end{lstlisting}

x86 Assembler
\begin{lstlisting}
        cmp ecx, edx
        je ifEnd ; "Jump if equal", Sprung wenn ecx == edx
                 ; ansonsten wird der Code ausgeführt
        ...
ifEnd:  ; Sprungadresse wenn ecx und edx
        ; gleich sind
\end{lstlisting}
\end{frame}


\begin{frame}{Stack-Operationen}

\begin{center}
\textbf{Stack}
\end{center}

\begin{itemize}
    \item analog zur Stack-Datenstruktur

    \item Operationen
    \begin{description}
        \item [PUSH] Hinzufügen von einem Element

        \item [POP] Löschen von einem Elementen
    \end{description}

    \item \textit{Stack Pointer} (ESP) zeigt auf das oberste Element

    \item ESP wächst nach unten (zu den niedrigeren Adressen)
\end{itemize}
\end{frame}


\begin{frame}[fragile]{Stack-Operationen}

\begin{center}
\textbf{Beispiel - PUSH}
\end{center}

\begin{lstlisting}
push    10  ; Speichert den Wert 10 auf dem Stack
\end{lstlisting}

\begin{multicols}{2}
\begin{minipage}{5cm}
\emph{Stack vor der Operation}\\
\begin{tabular}{c|c|}
    \cline{2-2}
   0xFF & 42\\ \cline{2-2}
   0xFE & \\ \cline{2-2}
   0xFD & \\ \cline{2-2}
        & ... \\ \cline{2-2}
   0x00 & \\ \cline{2-2}
\end{tabular}
ESP: 0xFF
\end{minipage}

\begin{minipage}{5cm}
\emph{Stack nach der Operation}\\
\begin{tabular}{c|c|}
    \cline{2-2}
   0xFF & 42\\ \cline{2-2}
   0xFE & 10\\ \cline{2-2}
   0xFD & \\ \cline{2-2}
          & ... \\ \cline{2-2}
     0x00 & \\ \cline{2-2}
\end{tabular}
ESP: 0xFE
\end{minipage}
\end{multicols}
\end{frame}


\begin{frame}[fragile]{Stack-Operationen}

\begin{center}
\textbf{Beispiel - POP}
\end{center}

\begin{lstlisting}
pop    eax  ; Speichert das zuletzt hinzugefügte Element
             ; im Stack in das Register eax
\end{lstlisting}

\begin{multicols}{2}

\begin{minipage}{5cm}
\emph{Stack vor der Operation}
\begin{tabular}{c|c|}
    \cline{2-2}
   0xFF & 42\\ \cline{2-2}
   0xFE & 10\\ \cline{2-2}
   0xFD & \\ \cline{2-2}
        & ... \\ \cline{2-2}
     0x00 & \\ \cline{2-2}
\end{tabular}
ESP: 0xFE
\end{minipage}

\begin{minipage}{5cm}
\emph{Stack nach der Operation}
\begin{tabular}{c|c|}
    \cline{2-2}
   0xFF & 42\\ \cline{2-2}
   0xFE & 10\\ \cline{2-2}
   0xFD & \\ \cline{2-2}
          & ... \\ \cline{2-2}
     0x00 & \\ \cline{2-2}
\end{tabular}
ESP: 0xFF
\end{minipage}
\end{multicols}
\end{frame}


\begin{frame}[fragile]{Stack-Operationen}

\begin{center}
\textbf{Funktionen / Prozeduren}
\end{center}

Nutzen \begin{itemize}
    \item Wiederverwendbarkeit
    \item Isolation von Quellcode
\end{itemize}

\makebox{}

Funktionen lassen in drei Teilschritte zerlegen
\begin{enumerate}
    \item Eintreten in die Funktion (Subroutine)
    \item Abarbeitung der Funktion
    \item Verlassen der Funktion (Subroutine)
\end{enumerate}
\end{frame}

\begin{frame}[fragile]{Stack-Operationen}

\begin{center}
\textbf{Eintreten in die Funktion - call-Mnemonic}
\end{center}

Syntax
\begin{lstlisting}
call    Adresse
\end{lstlisting}

Als Adresse wird meistens ein Label benutzt, dass vom Compiler dann automatisch durch die richtige Adresse ersetzt wird.

\makebox{}

Ablauf des \texttt{call}-Mnemonic
\begin{enumerate}
    \item Speichern des aktuellen \textit{Instruction Pointers} auf dem Stack
    \item Laden des neuen \textit{Instruction Pointers} in das Register \texttt{EIP}
    \item Beginn der Abarbeitung der Subroutine
\end{enumerate}

\end{frame}


\begin{frame}[fragile]{Stack-Operationen}

\begin{center}
\textbf{Verlassen der Funktion - ret-Mnemonic}
\end{center}

Syntax
\begin{lstlisting}
ret    [AnzahlDerRückgabeParameter]
\end{lstlisting}

\makebox

\begin{enumerate}
    \item \textit{Instruction Pointer} in das \texttt{EIP} Register laden
    \item Modifizierung des \textit{Stack Pointers} je nach Anzahl der zurückgebenden Parametern.
    \item Fortführung des Hauptprogramms
\end{enumerate}
\end{frame}


\begin{frame}[fragile]{Stack-Operationen}

\begin{center}
\textbf{Beispiel}
\end{center}

\begin{lstlisting}
myFunc: mov    eax, 10 ; Speichert im Register eax den Wert 10
        ret            ; Stellt den IP wieder her
                       ; und springt somit zurück zum Aufrufer

main:   call myFunc    ; Ruft die Subroutine myFunc auf
\end{lstlisting}
\end{frame}
