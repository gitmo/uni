\section{Geschichte}

\subsection{Intel Corporation in den 70er Jahren}
\begin{frame}{Geschichte}
	\begin{center}
		\textbf{Intel Corporation in den 70er Jahren}
	\end{center}
	\begin{itemize}
		\item eine von vielen Chip-Herstellern
		\item starke Konkurrenz mit Zilog Incorporation und seinem Z80 Mikrocontroller
		\item vorherrschende Word-Größe war 8 Bit
		\item Verzögerung der neuen Prozessorgeneration von Intel (32 Bit)
	\end{itemize}
\end{frame}

\subsection{Intels Antwort auf den Z80}
\begin{frame}{Geschichte}
	\begin{center}
		\textbf{Intels Antwort auf den Z80}
	\end{center}
	\begin{itemize}
		\item 8086 16 Bit Mikroprozessor
	 	\item entwickelt von einem Team um Stephen Morse 
		\item Urvater der 80x86-Familie 
		\item sehr langsam im Vergleich zu anderen 16 Bit Mikroprozessoren
	\end{itemize}

	$\rightarrow$ bedingt ein Erfolg
\end{frame}

\subsection{PC-Revolution}
\begin{frame}{Geschichte}
	\begin{center}
		\textbf{IBMs \textit{Personal Computer}	1981}
	\end{center}

	\begin{itemize}
		\item \emph{Ziel}: preisgünstiger Computer
		\item basierend auf Intels 8088-Mikroprozessor
		\item Aufrüstbar mittels Steckkarten
		\item offene Lizenzpolitik
		\item MS-DOS als Betriebssystem
	\end{itemize}

	$\rightarrow$ Kassenschlager und Grundstein für den Siegeszug der x86-Architektur 
\end{frame}

\begin{frame}{Geschichte}
	\begin{center}
		\textbf{Zitate}
	\end{center}

	\begin{block}{Stephen Morse}
		Any bright engineer could have designed the processor. It would probably have had a radically different instruction set, but all PCs today would be based on that architecture instead.
	\end{block}

	\pause

	\begin{block}{Bradley}
		If IBM had chosen the Motorola 68000 for the IBM PC (as some wanted), we would have had the WinOla duopoly rather than the Wintel duopoly
	\end{block}
\end{frame}
