\section{Geschichte}

\begin{frame}{Geschichte}
	\textbf{Intel Corporation in den 70er Jahren}
	\begin{itemize}
		\item eine von vielen Chip-Herstellern
		\item starke Konkurrenz mit Zilog Incorporation und seinem Z80 Mikrocontroller
		\item vorherrschende Word-Größe war 8 Bit
		\item Verzögerung der neuen Prozessorgeneration von Intel (32 Bit)
	\end{itemize}
\end{frame}

\begin{frame}{Geschichte}
	\textbf{Intels Antwort auf den Z80}
	\begin{itemize}
		\item 8086 16 Bit Mikroprozessor
	 	\item entwickelt von einem Team um Stephen Morse 
	\end{itemize}

	$\rightarrow$ bedingt ein Erfolg
\end{frame}

\begin{frame}{Geschichte}
	\textbf{IBMs \textit{Personal Computer}	1981}
	\begin{itemize}
		\item Ziel: preisgünstiger Computer
		\item basierend auf Intels 8088-Mikroprozessor
	\end{itemize}

	$\rightarrow$ Kassenschlager und Grundstein für den Siegeszug der x86-Architektur 
\end{frame}

\begin{frame}{Geschichte}
	\begin{quote}
	Any bright engineer could have designed the processor. It would probably have had a radically different instruction set, but all PCs today would be based on that architecture instead.
	\end{quote}

	\textit{Stephen Morse}
\end{frame}

\begin{frame}{Geschichte}
	\begin{quote}
	If IBM had chosen the Motorola 68000 for the IBM PC (as some wanted), we would have had the WinOla duopoly rather than the Wintel duopoly
	\end{quote}
	\textit{Bradley (IBM)}
\end{frame}
