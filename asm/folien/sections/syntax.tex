\section{Syntax}

\begin{frame}{Syntax}
\begin{center}
\textbf{Grundlegende Syntax}
\end{center}

  Syntax
  \texttt{[label:] mnemonic [argument1][, argument2][, argument3]}

  \makebox{}

  \begin{description}
    \item [label] Sprungmarke (optional)
    \item [mnemonic] Instruktionssymbol
    \item [argument] Parameter (optional, maximal 3)
  \end{description}

  \makebox{}

  jede Zeile entspricht einer Prozessorinstruktion
\end{frame}


\begin{frame}{Syntax}
  \begin{center}
  \textbf{synaktische Unterschiede zwischen Intel und AT\&T}
  \end{center}

  \begin{small}
  \begin{table}[h]  % place here\usepackage[cm]{fullpage}
  \begin{tabular}{lll}
  \\                            & INTEL SYNTAX                  & AT\&T SYNTAX
  \\\hline
  \\  Parameter                 & \tt mnem dest, src, const     & \tt mnem src, dest, const
  \\  Adressierung              & \tt [base+index*scale+disp]   & \tt disp(base, index, scale)
  \\  Register                  & \tt eax                       & \tt \%eax
  \\  Konstante                 & \tt 0xFF                      & \tt \$0xFF
  \\  Dereferenzieung           & \tt [addr]                    & \tt addr(,1)
  \\  Absolute Adresse          & \tt addr                      & \tt *addr
  \\  {\tt byte} Instruktion    & \tt mov byte ptr              & \tt movb
  \\  {\tt word} Instruktion    & \tt mov word ptr              & \tt movw
  \\  {\tt dword} Instruktion   & \tt mov dword ptr             & \tt movl
  \end{tabular}
  \end{table}
  \end{small}

  \begin{multicols}{2}
    \begin{minipage}{5cm}
      \begin{center}
      \textbf{Intel}\\
        \texttt{mov    eax, 5}
      \end{center}
    \end{minipage}

    \begin{minipage}{5cm}
      \begin{center}
      \textbf{AT\&T}\\
        \texttt{movw   \$5, \%eax}
      \end{center}
    \end{minipage}
  \end{multicols}

\end{frame}
