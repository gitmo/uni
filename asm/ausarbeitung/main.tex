\documentclass[11pt,german]{scrartcl}

% See geometry.pdf to learn the layout options.
\usepackage{geometry}
\geometry{a4paper}

% To begin paragraphs with an empty line
\usepackage[parfill]{parskip}

% Use utf-8 encoding for foreign characters
\usepackage[utf8]{inputenc}

% Setup for fullpage use
\usepackage{fullpage}

% use multicol package
\usepackage{multicol}

% use multirow package
\usepackage{multirow}

% Running Headers and footers
\usepackage{fancyhdr}
\pagestyle{fancyplain}

\usepackage{url}
\usepackage{makeidx}
\usepackage[colorlinks=true, linkcolor=blue, urlcolor=blue]{hyperref}

% Surround parts of graphics with box
\usepackage{boxedminipage}

% Package for including code in the document
\usepackage{listings}
\usepackage{color}
\usepackage{moreverb}

% Alternative monospaced font
\usepackage[T1]{fontenc}
\usepackage[scaled]{beramono}


\title{Intel x86 Assembler\\\small{Proseminar Programmiersprachen}}
\author{Sebastian Raitza, Nico von Geyso}

\makeindex

\begin{document}

\maketitle

\input{sections/abstract.tex}

\tableofcontents

\section{Einführung}
Während vor einigen Jahren das Wissen über Assembler-Programmierung unabdingbar war, ist es heute ein leichtes Unterfangen gar komplexe Probleme in einer Hochsprache zu lösen ohne jemals hardwarenah programmiert zu haben.
Der Artikel versucht die geschichtliche Entwicklung der x86-Architektur und seine Besonderheiten knapp darzustellen.
Neben Syntax und Semantik des x86 Assemblers wird auch auf das für die heutige Zeit wichtige Thema des \textit{Reverse Engineering} eingegangen.


\section{Geschichte}

Die Dominanz der x86-Architektur war vor 30 Jahren keineswegs vorauszusehen.
Ende der 70er Jahre war die Firma Intel eine von vielen Chip-Herstellern in der westlichen Welt.
Desweiteren hatte Intel mit Zilog Incorporation, einer Firma ehemaliger Intel Mitarbeiter, einen starken aufstrebenden Konkurrenten.
Dieser machte ihnen mit ihrem 8080-Mikroprozessor-Klon Z80 stark zu schaffen. 

Die Hoffnung mit der neuen Prozessorgeneration in Form des iAPX 432 einen Vorteil gegenüber der starken Konkurrenz zu gewinnen, musste immer wieder verschoben werden.
Der Wechsel auf eine 32-Bit-Architektur schien für die damalige Chipindustrie eine größere Herausforderung zu sein als erwartet und so traten stets neue Probleme auf.

Um den Anschluß nicht ganz zu verlieren, versuchte Intel mit einem kleinen Team um den Software-Entwickler Stephen Morse eine Antwort auf den damals populären Z80 zu finden.
Der von Stephen Morse entwickelte 8086 16Bit-Mikroprozessor war dennoch nur bedingt ein Erfolg.
Ändern sollte sich dies erst, als IBM 1981 den sogenannten preisgünstigen \textit{Personal Computer} entwickelte. Das IBM Model 5150 war ein Kassenschlager und läutete die PC-Revolution in Unternehmen und später auch im privaten Bereich ein.
Verbaut war hier ein 8088-Mikroprozessor von Intel. Dieser war der Nachfolger des 8086-Mikroprozessor und verwendete somit die gleiche Instruktionsmenge.
Hierdurch wurde der Grundstein für den bis heute anhaltenden Siegeszug der x86-Architektur gelegt.\cite{pcworld} 

Ein Grund für die heute Monopolstellung der x86-Architektur ist seine Abwärtskompatibilität.
Jedes für den 8086-Prozessor geschriebene Programm kann noch heute ohne Modifizierungen auf der neusten x86-Prozessorgenerations ausgeführt werden.
Umgesetzt wird dies mit Hilfe sogenannter \textit{Operations modes}. Zu dem anfänglichen \textit{Real mode} des 8086-Prozessors, kamen mit der Zeit immer weitere hinzu.
Hierdurch konnte unter anderem der addressiebare Addressraum erweitert, virtueller und geschützter Speicher eingeführt werden.
Die Word-Größe des x86-Assemblers wurde im Laufe der Zeit zweimal vergrößert. Das erste Mal im Jahre 1985. Hier wurde von Intel der Wechsel von 16 zu 32 Bit vollzogen. In den Jahren 1999 bis 2003 erweiterte AMD die bis dato 32-Bit x86-Architektur auf 64 Bit.

Die x86-Architektur lässt sich ursprüglich der CISC\footnote{\textit{complex instruction set computer}}-Architektur zu ordnen.
Neueste Prozessorgenerationen bringen jedoch diese Eindeutigkeit ins wanken, da hier der Prozessor selbst gewisse Befehle auf eine RISC\footnote{\textit{reduced instruction set computer}}-Architektur  reduziert.
Aufgrunddessen wird in manchen Kreisen mittlerweile auch von der sogenannten \textit{POST-RISC}-Archtiktur gesprochen.\cite{postrisc}


\section{Syntax}

In der Welt des x86-Assemblers gibt es zwei große Syntaxfamilien: die Intel und
die AT\&T-Syntax. Während erstere fast ausschließlich in einer Vielzahl von Assemblern (NASM, TASM, MASM, YASM, usw.) und Disassemblern (\emph{IDA Pro, OllyDbg}\cite{disasm}) zum Einsatz kommt
und Intel die x86-Platform damit dokumentiert, ist letztere dennoch nicht
gänzlich irrelevant. Auf UNIX- und Linux-Betriebssystemen wurde lange Zeit nur AT\&T's
Syntaxstil von den mitgelieferten Assemblern \texttt{as} bzw. \texttt{gas}
unterstützt. Insbesondere durch die Verbreitung der \emph{GNU Compiler Collection}
hat sich dieser Stil bis heute gehalten, obwohl der \emph{GNU Assembler} mittlerweile
auch die Intel-Syntax beherrscht.

Überhaupt sind die Unterschiede beider Stile gleichmächtig und lassen sich
trotz erheblicher Unterschiede im Erscheinungsbild problemlos ineinander konvertieren
(z.B. mit \emph{Intel2GAS} \cite{i2g}). Für die folgenden Beispiele wird die
Intel-Syntax als die kanonische Referenz betrachtet und jeweils der
äquivalenten AT\&T-Syntax gegenübergestellt.

\subsection{Arbeitsweise des Assemblers}

Beim Programmieren mit einem Assembler beschreibt jede Zeile genau eine
Prozessorinstruktion über ihr Symbol, dem sogenannten Mnemonik, gefolgt von
den dazugehörigen optionalen Argumenten. Die Aufgabe des Assemblers ist es nun,
dieses symbolische Programm direkt in die entsprechende Folge binären
x86-Maschinencode – sogenannten Opcodes, die üblicherweise byteweise,
hexadezimal dargestellt werden – zu übersetzen.

Vereinfacht betrachtet schlägt der Assembler dazu in einer großen Symboltabelle nach in der für jedes Tupel von Mnemonik und Argumenten der entsprechende Opcode vermerkt
ist. Der Programmierer könnte zwar auch direkt in Opcodes programmieren, bekommt
aber durch den Assembler die Hilfestellung sich statt hexadezimalen Code
Symbole merken zu dürfen.

\subsection{Instruktionen}

Eine Zeile in einem Assemblerprogramm hat folgende Form, die in beiden
Syntaxstilen identisch ist. Zwischen Groß- und Kleinschreibung wird nicht unterschieden.

\texttt{[label:] mnemonic [argument1][, argument2][, argument3]}

Label ist eine Sprungmarke und optional, ebenso wie die Argumente. Deren
erforderliche Anzahl hängt von der Instruktion ab. Maximal sind drei, häufiger zwei
üblich. Das Mnemonik bzw. Instruktionssymbol ist eine kurze
Zeichenkette, wie zum Beispiel {\tt MOV, ADD} oder {\tt PUSH}, und abstrahiert gleich über eine ganze Klasse von Opcodes mit derselben Funktion.\cite{intelmanual} So ist zum
Beispiel die Kopieroperation für Argumente verschiedener Art (Register,
Konstanten) immer das Mnemonik {\tt MOV} obwohl sich die Opcodes unterscheiden.

\subsection{Parameter}

Bei vielen Befehlen sind zwei Parameter üblich. Es wird dann meist von Quell-
und Zieloperanden gesprochen (source and destination parameter). Die Semantik
der Parameterreihenfolge der Intel-Syntax ist entgegengesetzt zur AT\&-Syntax.

Ein Parameter kann von dreierlei Art sein: ein Register, eine Konstante oder
eine Speicheradresse. Die Registerbezeichnungen (z.B. {\tt EAX, EBP}) sind bei Intel und
AT\&T identisch, allerdings erfordert letztere noch ein Prozentzeichen im
Prefix. Konstanten erhalten bei AT\&T ein Dollarzeichen-Präfix. Intel-Assembler
benötigen diese Kennzeichnung nicht.

\hspace{5mm} \makebox[1.5cm]{Intel: \hfill}
\texttt{mov eax, 5}

Eine weitere Eigenart der AT\&T-Syntax ist ein Suffix für Befehle (siehe
Tabelle~\ref{tab:syntaxdiffs}), der die Parametergröße beschreibt: Byte, Word
(16 Bit), Long oder Double-Word  (32 Bit) bzw.  Quad-Word (64-Bit). Für die
Syntax von Intel leitet der Assembler den Typ automatisch vom Parameter ab.

\hspace{5mm} \makebox[1.5cm]{AT\&T: \hfill}
\texttt{movl \$5, \%eax}

Ein Parameter kann eine absolute Adresse enthalten oder indirekt auf den Inhalt
einer Speicherstelle zeigen. Die Speicheradresse für die Dereferenzierung kann
aus einem Ausdruck berechnet werden.  Diese effektive Adresse wird bei Intel
aus Variablen in eckigen Klammern gebildet und kann eine Typangabe wie
\texttt{BYTE},  \texttt{WORD} (2 Byte) und \texttt{DWORD} (4 Byte) gefolgt von  \texttt{PTR}
(Pointer) vorangestellt bekommen.

\hspace{5mm} \makebox[1.5cm]{Intel: \hfill}
\texttt{mov eax, dword ptr [ebp+4]}

In AT\&T Syntax berechnet sich eine Adresse nach dem sogenannten \emph{base
indexed addressing}-Schema aus den Einzelkomponenten Verschiebung (auch
\emph{displacement}), Basis, Index und Skalierung wie folgt: $Verschiebung + Basis +
Index*Skalierung$ . Ein Beispiel:

\hspace{5mm} \makebox[1.5cm]{AT\&T: \hfill}
\texttt{movl 4(\%ebp), \%eax}

Die wesentlichen Syntax-Unterschiede sind in der Tabelle~\ref{tab:syntaxdiffs}
zusammengefasst.

%\clearpage

\begin{table}[ht]  % place here
\begin{tabular}{lll}
\\                            & INTEL SYNTAX                  & AT\&T SYNTAX
\\\hline
\\  Parameter                 & \tt mnem dest, src, const     & \tt mnem src, dest, const
\\  Adressierung              & \tt [base+index*scale+disp]   & \tt disp(base, index, scale)
\\  Register                  & \tt eax                       & \tt \%eax
\\  Konstante                 & \tt 0xFF                      & \tt \$0xFF
\\  Dereferenzierung          & \tt [addr]                    & \tt addr(,1)
\\  Absolute Adresse          & \tt addr                      & \tt *addr
\\  {\tt byte} Instruktion    & \tt mov byte ptr              & \tt movb
\\  {\tt word} Instruktion    & \tt mov word ptr              & \tt movw
\\  {\tt dword} Instruktion   & \tt mov dword ptr             & \tt movl
\end{tabular}
\caption{Syntaktische Unterschiede} \label{tab:syntaxdiffs}
\end{table}

\subsection{Assembler-Direktiven}
Jeder Assembler besitzt einen eigenen Satz an spezifischen Direktiven. Streng
genommen gehören diese nicht zur Syntax, allerdings sind sie nötig um einige
Meta-Informationen zur Code-Struktur zu definieren. Die Direktiven sind stark
abhängig von der eingesetzten Software (\emph{NASM, YASM, GNU as}, etc.) und selten
herrscht Kompatibilität unter den Assembler-Dialekten. Fast immer gibt es aber ähnlich lautende Entsprechungen.

Eine Typische Anweisung ist die Zuordnung eines Bezeichners zu einem Code- oder
Datensegment. Die Adressen der Instruktionen werden beim Assemblieren ihrem
Segment zugeordnet. Bei \emph{NASM} heißt die Direktive \texttt{section}
bei \emph{GAS} \texttt{.section}. Die Sektionen \texttt{.text} und
\texttt{.data} für das Code- bzw. statische Datensegment sind dabei traditionell
unter Unix-artigen Systemen gebräuchlich. Die Bezeichner können aber
beliebig gewählt werden.

Oft ist es auch nötig Datenbereiche zu reservieren oder statisch zu
initialisieren, zum Beispiel um feste Strings zu definieren. Dazu dienen
Pseudo-Instruktionen wie \texttt{DB, DW, DD}, etc. (\emph{define byte, define
word, define dword}) und \texttt{RESB, RESW, RESQ} (\emph{reserve byte}, usw.)
bei \emph{NASM} und auch \emph{MASM}. Die Definition von Konstanten erfolgt mittels
\texttt{EQU}. Diesen Definitionen kann ein Label vorangestellt werden, über
welches die Adresse der jeweiligen Daten im Code referenziert wird. Beim
Assemblieren werden diese Labels in konkrete Adressen übersetzt.



%Register
%Adressierung
%Register
%Adressierung
\section{Architektur}
x86-Assembler umfasst eine Instruktionsmenge die ursprünglich für Intels 8086 CPU konzipiert war. Diese basiert auf einer "von Neumann"-Architektur. Hier wird, im Gegensatz zur Harvard-Architektur, für Daten und Programme der gleiche Speicher verwendet. Es existierte ein fester Registersatz von 16 Registern der sich im Laufe der Zeit immer weiter vergrößert hat. Ursprünglich hatte x86 vier general-purpose Registern (ax,bx,cx und dx), vier Segmentierungsregistern zur Speicheradressierung (cs, ds, es und ss), sowie FLAG-Registern und dem Instruktionszeiger-Register.


\section{Instruktionen}

\subsection{Daten kopieren}
%Introduction to 80x86 Assembly Language and Computer Architecture - Seite 86
Einer der am häufigsten verwendeten Operationen ist das Kopieren von Daten von einer Stelle zu einer anderen.
Hierzu wird im x86-Assembler das \texttt{mov}-Mnemonic verwendet. Es hat die folgende Syntax:

\begin{verbatim}mov    Zieladresse, Quelle\end{verbatim}

Es können immer nur Daten der gleichen Größe kopiert werden.
Das heißt es muss sichergestellt sein, dass die Register beziehungsweise Adressen dementsprechend gleich große Werte speichern können.
Im folgendem Beispiel wird in das Register eax der Wert aus dem Register ebx kopiert:

\begin{verbatim}mov    eax, ebx\end{verbatim}

Um einer Zieladresse einen festen Wert zu zuweisen kann statt einer Adresse im zweiten Parameter auch eine Konstante stehen:

\begin{verbatim}mov    eax, 5\end{verbatim}

Hier wird dem Register \textit{eax} der Wert 5 zugewiesen.

Wichtig ist, dass es sich bei einer der beiden Adressen um ein Register handeln muss.


\subsection{einfache mathematische Operationen}
%inc,dec,add,sub,xor
%Introduction to 80x86 Assembly Language and Computer Architecture - Seite 96
Der x86-Assembler verfügt unter anderem über folgende mathematische Operationen

\begin{description}
	\item [INC] Inkrementiert das Ziel
		\begin{verbatim}INC Ziel\end{verbatim}

	\item [ADD] Addiert Ziel und Quelle und speichert das Ergebnis in Ziel
		\begin{verbatim}INC Ziel, Quelle\end{verbatim}

	\item [DEC] Dekrementierung  das Ziel
		\begin{verbatim}INC Ziel\end{verbatim}

	\item [SUB] Subtrahiert Quelle von Ziel und speichert das Ergebnis in Ziel
		\begin{verbatim}INC Ziel, Quelle\end{verbatim}
\end{description}

In den sogenannten \textit{Flag}-Registern kann überprüft werden ob das Ergebnis negativ oder null ist, sowie festgestellt werden ob es einen Überlauf gab.

\subsection{jmp, je ,jne}

\subsection{cmp}

\subsection{push/pop und call/leave}


\section{Hello World}

In diesem Abschnitt soll ein einführendes Beispiel zur Programmieren in x86-Assembler gegeben werden. Um in der Tradition von Programmiersprachen-Literatur zu bleiben, wird hier ein kurzes Programm demonstriert, das ``Hello, World!'' auf der Konsole ausgibt.

Assembler-Programme sind naturgemäß stark Plattformabhängig, da auf niedrigster Ebene mit dem Betriebssystem kommuniziert wird.
Das folgende Listing ist für \emph{Mac OS X} entwickelt worden, und
funktioniert auch auf den nah verwandten BSD-Unix-Systemen. Mit kleinen Änderungen kann es für Linux angepasst werden, die in einem extra Abschnitt beschrieben werden.

\subsection{Assemblieren und Binden}

Das Programm lässt sich mit folgender Shell Befehlen übersetzen und ausführen.

\begin{lstlisting}[caption=Assemblieren und Binden von hello.asm]
$ nasm -f macho32 hello.asm
$ ld hello.o -o hello
$ ./hello
Hello, World!
\end{lstlisting}

Der Assembler NASM (Netwide Assembler) übersetzt das Listing in Objektcode in 32-bit Mach-O-Format – dabei handelt es sich um das Dateiformat für ausführbaren Code unter \emph{Mac OS X}. Das Ergebnis ist der Zwischencode in \texttt{hello.o}. Im zweiten Schritt wird der Linker \texttt{ld} angewiesen eine ausführbare Programmdatei (\emph{ Executable}) mit dem Namen \texttt{hello} zu erzeugen, welche im Anschluss ausgeführt wird. Die Vorgehensweise beim Binden ist analog zur Arbeit mit dem C-Compiler von \emph{GCC} der ebenfalls im Zwischenschritt Objektcode-Dateien aus C-Programmen erzeugt.

\subsection{Code für BSD-Unix Systeme}

Im Folgenden soll das Programm genauer erläutert werden. Genaugenommen handelt es sich nur bei den Zeilen ab dem Label \texttt{kernel} um Assembler-Instruktionen. Die vorhergehenden Zeilen sind Präprozessor-Makros (Prefix \%) und Assembler-Direktiven. Der Linker benutzt das Label {\tt start} als Einsprungspunkt in das \emph{Executable} (Linux: {\tt \_start}).
Dieses Label wird mit der Direktive {\tt global} als nach außen sichtbar markiert.

\begin{lstlisting}[numbers=left,caption=hello.asm]
; Preprocessor macros
%define stdout  1

%define SYS_exit    1       ; System calls
%define SYS_write   4

; segment for static data
section .data

message   db 'Hello, World!', 0Ah  ; End with newline
length    equ $ - message          ; Length = current - previous address

; code segment
section .text
global start                ; Make start function externally visible for linker

kernel:                     ; Expects syscall number in eax register
    int     80h             ; Call kernel
    ret

start:                      ; entry symbol on Mac OS X

    push    length          ; size_t nbyte
    push    message         ; const void *buf
    push    stdout          ; int filedescriptor
    mov     eax, SYS_write  ; Make the system call to write
    call    kernel

    add     esp, 12         ; 3 args * 4 bytes each = 12 bytes

    push    0               ; exit status returned to the operating system
    mov     eax, SYS_exit   ; Make the syscall to exit
    call    kernel
\end{lstlisting}

Das Listing besteht aus den Hauptprogramm, gekennzeichnet mit dem Label \texttt{start} und einem Unterprogramm mit dem Label \texttt{kernel}. Zur Ausgabe der Textnachricht wird hier direkt auf den System-Call \texttt{write} des Betriebssystemkernels zurückgegriffen.

Das Hauptprogramm bereitet die Parameter für {\tt write} vor, in dem es sie nach C-Aufruf"-konvention von hinten nach vorne mittels {\tt push} auf dem Stack ablegt. Der Stackpointer {\tt esp} wird mit jedem Push um 4 Bytes ($=$ 32 Bit) verringert -- zur Erinnerung: der Stack wächst von den hohen zu niedrigen Speicheradressen. Welche Parameter der System-Call benötigt beschreibt die Manual-Page {\tt man 2 write}:

\begin{lstlisting}
    ssize_t write(int fildes, const void *buf, size_t nbyte);
\end{lstlisting}

\begin{enumerate}
\item {\tt length}: Länge der Nachricht – eine Konstante definiert durch die
NASM Anweisung {\tt equ} als Abstand in Bytes Zwischen dem Label length und message
\item {\tt message}: Die Adresse der Nachricht definiert durch {\tt db} als String von ASCII-Bytes
\item {\tt stdout}: Der Dateideskriptor der Standardausgabe (=1) unter Unix-artigen Systemen
\end{enumerate}

Der Kernel wird mit Hilfe des Software-Interrupts 80h (hexadezimal 80, identisch unter Linux, BSD und OS X) angesprungen. Einen ganzzahliger Wert im Register \texttt{eax} dient als Angabe welcher System-call verwendet werden soll. Für {\tt write} ist das die Konstante 4. Nach Behandlung des Interrupts wird mittels \texttt{ret} an die nächste Instruktion im Hauptprogramm zurückgekehrt an der vorher mittels \texttt{call kernel} der Prozeduraufruf stattfand.

Der Stack enthält nun immer noch die drei Parameter an der Stelle des Stackpointers ({\tt esp}), die durch das Addieren von 12 Bytes -- jeder Parameter belegte 4 Bytes -- freigegeben werden können. Da das Programm aber als nächstes mit dem Syscall \texttt{exit} beendet wird, ist dies optional, da der Stack dann auch vom Betriebssystem freigegeben wird.


\subsection{Anpassungen für Linux}

Wie bereits erwähnt wurde sind beim Aufruf von Betriebssystemfunktionen festen Konventionen \cite{wp:callconv} der Parameterübergabe einzuhalten. Diese unterscheiden sich von System zu System (definiert in der ABI \cite{wp:abi}) und sogar von einem Compiler zum Anderen.

Der Linux-Kernel erwartet die sogenannten \emph{fastcall}-Konvention, wobei die Parameter über festgelegte Register übergeben werden, in sofern es deren Typ und Anzahl erlaubt. Dies ist oft effizienter, da es die Anzahl der teueren Speicherzugriffe auf den Stack minimiert. Prozessorregister arbeiten erheblich schneller als Cache oder Hauptspeicher. Das erste Argument wird dazu in {\tt ebx}, das zweite in {\tt ecx}, usw. übergeben.

\begin{lstlisting}[caption=Linux-Kernel Aufrufkonvention]
    ; Linux: fastcall convention
    mov     edx, length
    mov     ecx, message
    mov     ebx, stdout
    mov     eax, SYS_write  ; Make the system call to write
    call    kernel
\end{lstlisting}

Die zweite kleine Änderung betrifft das Label {\tt start}. Der Linker sucht unter Linux den Einsprungspunkt stattdessen bei {\tt \_start}.


\section{Reverse Engineering}

Assembler als Haupt-Programmiersprache für Softwareprojekte hat auf Grund der großen Verbreitung einer Vielzahl von Hochsprachen, mit samt deren Vorteilen wie höheres Abstraktionsnieveau, umfangreiche Werkzeuge, hohe Portierbarkeit und Wirtschaftlichkeit von Wartung und Kollaboration in Teams, stark an Bedeutung verloren. Auch sind die Hauptstärken von Assemble Schnelligkeit und geringe Codegröße durch das Aufkommen hochoptimierender Compiler und dem exponentiellem Wachstum von Prozessorleistung und Speicherkapazitäten in den Hintergrund getreten.

Doch auch ohne das heute noch viel direkt in Assembler programmiert wird ist die Kenntnis der Sprache unverzichtbar für das sogenannte \emph{Reverse Engineering}. Damit bezeichnet man im Software-Kontext die Disassemblierung und Analyse von binärem Programmcode, also allem übersetzten, ausführbaren Code eines Computers. Man kann es sich als ein "Zurückgehen im Entwicklungsprozess" vorstellen.\cite{Warden1992}

Dies kann aus vielerlei Motivation möglich sein:
\begin{itemize}
\item Debugging von Compilerfehlern oder Programmen ohne Quelltext
\item Schadcode-Analyse z.B. in der Antivirenindustry
\item Forensische Analyse nach einer Sicherheitskompromitierung ("Hacker-Angriff")
\item Sicherheitslücken suchen und aufdecken durch Sicherheitsforscher
\item Umgehung von Kopierschutz durch Entwickeln von \emph{Cracks} (binäre Patches)
\item Umgehung von Kopierschutz durch \emph{Keygeneratoren} – Analyse und Nachbau des Lizensierungsschemas einer kommerziellen Anwendung
\item Analyse, Umgehung oder Entfernung von \emph{Digital Rights Management}
\end{itemize}

Die Vorgehensweise soll am Beispiel der Schadcode-Analyse kurz beschrieben werden: ein Antivirenhersteller und unabhängiges Forscherteam enthält unbekannten Virencode durch Einsendung eines Tipgebers dem Einsammeln von infizierten Dateien auf speziellen "Locksystemen" (\emph{Honey-pots}). Dieser auch als \emph{Sample} bezeichnete Binärcode wird in einem Disassembler geladen und zunächst statisch analysiert. Neben dem Assembler-Listing und vorinitialisierten Daten wie Strings und Tabellen, erhält man oft auch Informationen Programmelemente höherer Abstraktionslevel wie Structs, Klassen und deren Methoden und Funktions- und Variablennamen (wenn sie Zwecken des Ausführens oder Debuggens erforderlich waren bzw. absichtlich in das Programm kompiliert wurden).
Oft lassen sich auch Flussdiagramme und Aktivierungsbäume von Funktionsaufrufen automatisch generieren.

\begin{figure}[h]
  \begin{center}
	\includegraphics[width=0.85\textwidth]{IDA-pro-flowdiagram.png}
  \end{center}
  \caption{Flussdiagramm eines Programmsegments in IDA Pro}
\end{figure}

Da diese Disassembler oft auch auf Seiten der Virenentwicklers zum Grundwerkzeug gehören werden in der Praxis oft Verschleierungs- oder sogar Verschlüsselungs-Techniken eingesetzt, die statische Code-Analyse erschweren oder unmöglich machen. Die meisten Disassembler sind aber auch Debugger und erlauben die Analyse zur Laufzeit – bei Schadcode meist in einer gesicherte Laborumgebung. Ein \emph{Trace} durch den Programmablauf hinterlässt dann eine Spur des tatsächlich ausgeführten, aktiven Codes.

Das Umgehen von Software-Sicherungen zum Kopierschutz oder Passwortabfragen ist ein anderes Einsatzfeld von Reverse-Engineering-Techniken im legalen Graubereich – Patent-, Eigentums- und Urheberechtsgesetze werden oft verletzt. Es gibt jedoch auch Enthusiasten, die \emph{Cracking} aus "sportlichen" Gründen betreiben und von den facettenreichen Fähigkeiten fasziniert sind die dazu häufig nötig sind. Ein legaler Weg sind sogenannte \emph{Crackme}, kleine Programme die von anderen "\emph{Reversern}" geschrieben wurden um sich gegenseitig auszuprobieren oder herauszufordern.

Ein weiteres Gebiet in dem Assemblercode geschrieben und gelesen werden, muss ist das Entwickeln bzw. Analysieren von \emph{Exploits}. So bezeichnet man kurze Programme zum aktiven Ausnutzen von Sicherheitslücke. Meist geht es dabei darum subversiv, eigenen Code in fremde Computersysteme einzuschleusen und sich nicht-autorisierten Zugang zu verschaffen. Assembler taucht hier meist in Form von \emph{Shellcode} auf, Code der so heißt, da er in der Regel zum Starten einer privilegierten, fernsteuerbaren Shell auf dem Fremdsystem eingeschleust wird. Der Shellcode wird in Form einer mit falscher Absicht zusammengesetzten Datenangabe eingebracht, auf die entgegennehmende Software fehlerhaft reagiert. Der Shellcode befindet sich dann schon auf dem Stack oder Heap des anfälligen Programms und muss geschickt unter Ausnutzung der Sicherheitslücke angesprungen werden. Gute Assemblerkentnisse sind für den Exploit-Programmierer Voraussetzung. Die Motivationen sind wiederum vielschichtig und reichen von wissenschaftlicher Forschungsgeist über persönlichen Ehrgeiz bis zu kriminelle Absichten.


\section{Fazit}
Aus heutiger Sicht ist die Bedeutung von x86-Assembler stark zurückgegangen. Wurden vor 20 Jahren noch ein viele Programme hardwarenah programmiert, so hat sich dies mittlerweile stark verändert. Die Lesbarkeit und Wartbarkeit können nicht mit denen von aktuellen Hochsprachen wie Java oder Python mithalten. Dennoch ist ein Grundverständnis vom x86-Assembler wichtig, da letztendlich jede Hochsprache in Maschinencode übersetzt wird.
Große Bedeutung hat Assembler aber mittlerweile beim sogenannten Reverse Engineering erlangt....
Abschließend lässt sich sagen, dass Assembler das Latein des Computers ist. Lesen und Verstehen reichen für den Größteil der Probleme völlig aus.


\section{Quellen}

\begin{thebibliography}{9}

\bibitem{pcworld}
  \emph{Birth of a Standard: The Intel 8086 Microprocessor}
	von Benj Edwards
	in PC World (17. Juni 2008)
	%http://www.pcworld.com/article/146957/birth_of_a_standard_the_intel_8086_microprocessor.html

\bibitem{postrisc}
	\emph{Beyond RISC - The Post-RISC Architecture}
	von Mark Brehob, Travis Doom, Richard Enbody, William H. Moore, Sherry Q. Moore, Ron Sass, Charles Severance
	in IEEE Micro Mai 1996
	%http://www.cse.msu.edu/~enbody/postrisc/postrisc2.htm (IEEE Micro 3/96)

\end{thebibliography}


\end{document}

