\section{Geschichte}

Ende der 70er Jahre war die Firma Intel eine von vielen Chip-Herstellern in der westlichen Welt. Zilog Incorporation machte ihnen mit ihrem 8080-Klon Z80 stark zu schaffen. Bei Intels  neuen Prozessorgeneration in Form des iAPX 432 kamen immer wieder neue Probleme auf und so musste dieser desöfteren verschoben werden. Intels Antwort auf den damals populären Z80 sollte nun ein kleines Team um Stephen Morse finden.
Der von Stephen Morse entwickelte 8086 16Bit-Prozessor war nur bedingt ein Erfolg. Ändern sollte sich dies erst, als IBM den sogenannten preisgünstigen "Personal Computer" entwickelte. Das IBM Model 5150 war ein Kassenschlager und läutete die PC-Revolution in Unternehmen ein. Als Prozessor fand Intels 8088 Verwendung. Dieser verwendet die gleiche Assembler-Sprache wie sein älterer Bruder 8086. Stephen Morse ist somit einer der Urväter des x86-Assemblers.
Herausstechend ist die Abwärtskompatibilität der x86-Architektur. Jedes für den 8086-Prozessor geschriebene Programm kann auf der neusten x86-Prozessorgenerations ausgeführt werden. Umgesetzt wird dies mit Hilfe sogenannter "Operations modes". Zu dem anfänglichen "Real mode" des 8086-Prozessors, kamen mit der Zeit immer weitere hinzu. Hierdurch konnte unter anderem der addressiebare Addressraum erweitert, virtueller und geschützter Speicher eingeführt werden.
Die Word-Größe des x86-Assemblers wurde im Laufe der Zeit zweimal vergrößert. Das erste Mal im Jahre 1985. Hier wurde von Intel der Wechsel von 16 zu 32 Bit vollzogen. In den Jahren 1999 bis 2003 erweiterte AMD den ursprüglichen 32Bit x86-Assembler auf 64 Bit.
Der x86-Assembler lässt sich ursprüglich der CISC-Architektur zu ordnen. Neueste Prozessorgenerationen bringen diese Eindeutigkeit aber ins wanken, da hier der Prozessor selbst gewisse Befehle auf eine RISC-Architektur reduziert. Aufgrunddessen wird in manchen Kreisen mittlerweile auch von der sogenannten "POST-RISC"-Archtiktur gesprochen
