\section{Syntax}

In der Welt des x86-Assemblers gibt es zwei große Syntaxfamilien: die Intel und
die AT\&T-Syntax. Während erstere fast ausschließlich in einer Vielzahl von Assemblern (NASM, TASM, MASM, YASM, usw.) und Disassemblern (IDA Pro, OllyDbg) zum Einsatz 
kommt
und Intel die x86-Platform damit dokumentiert, ist letztere dennoch nicht
gänzlich irrelevant. In der UNIX und Linux-Welt wurde lange Zeit nur AT\&T's
Syntaxstil von den ausgelieferten Assemblern \texttt{as} bzw. \texttt{gas}
unterstützt. Insbesondere durch die Verbreitung der GNU Compiler Collection
hat sich dieser Stil bis heute gehalten, obwohl der GNU Assembler mittlerweile
auch die Intel-Syntax beherrscht.

Überhaupt sind die Unterschiede beider Stile gleichmächtig und lassen sich
trotz erheblicher optischer Unterschiede problemlos ineinander konvertieren
(z.B. mit Intel2GAS \cite{i2g}). Für die folgenden Beispiele wird die
Intel-Syntax als die kanonische Referenz betrachtet und jeweils der
äquivalenten AT\&T-Syntax gegenübergestellt.

\subsection{Arbeitsweise des Assemblers}

Beim Programmieren mit einem Assembler beschreibt jede Zeile genau eine
Prozessorinstruktion über ihr Symbol, dem sogenannten Mnemonik, gefolgt von
den dazugehörigen optionalen Argumenten. Die Aufgabe des Assemblers ist es nun,
dieses symbolische Programm direkt in die entsprechende Folge binären
x86-Maschinencode – sogenannten Opcodes, die üblicherweise byteweise,
hexadezimal dargestellt werden – zu übersetzen.

Vereinfacht betrachtet schlägt der Assembler dazu in einer großen Symboltabelle nach in der für jedes Tupel von Mnemonik und Argumenten der entsprechende Opcode vermerkt
ist. Der Programmierer könnte zwar auch direkt in Opcodes programmieren, bekommt
aber durch den Assembler die Hilfestellung sich statt hexadezimalen Code
Symbole merken zu dürfen.

\subsection{Instruktionen}

Eine Zeile in einem Assemblerprogramm hat folgende Form, die in beiden
Syntaxstilen identisch ist. Zwischen Groß- und Kleinschreibung wird nicht unterschieden.

\texttt{[label:] mnemonic [argument1][, argument2][, argument3]}

Label ist eine Sprungmarke und optional, ebenso wie die Argumente. Deren
erforderliche Anzahl hängt von der Instruktion ab. Maximal sind drei, häufiger zwei
üblich. Das Mnemonik bzw. Instruktionssymbol ist eine kurze
Zeichenkette, wie zum Beispiel {\tt MOV, ADD} oder {\tt PUSH}, und abstrahiert gleich über eine ganze Klasse von Opcodes mit derselben Funktion.\cite{intelmanual} So ist zum
Beispiel die Kopieroperation für Argumente verschiedene Art (Register,
Konstanten) immer das Mnemonik {\tt MOV} obwohl sich die Opcodes unterscheiden.

\subsection{Parameter}

Bei vielen Befehlen sind zwei Parameter üblich. Es wird dann meist von Quell-
und Zieloperanden gesprochen (source and destination parameter). Die Semantik
der Parameterreihenfolge ist bei Intel-Syntax entgegengesetzt zur AT\&-Syntax.

Ein Parameter kann von dreierlei Art sein: ein Register, eine Konstante oder
eine Speicheradresse. Die Registerbezeichnungen (z.B. {\tt EAX, EBP}) sind bei Intel und
AT\&T identisch, allerdings erfordert letztere noch ein Prozentzeichen im
Prefix. Konstanten erhalten bei AT\&T ein Dollarzeichen-Präfix. Intel-Assembler
benötigen diese Kennzeichnung nicht.

\hspace{5mm} \makebox[1.5cm]{Intel: \hfill}
\texttt{mov eax, 5}

Eine weitere Eigenart der AT\&T-Syntax ist ein Suffix für Befehle (siehe
Tabelle~\ref{tab:syntaxdiffs}), der die Parametergröße beschreibt: Byte, Word
(16 Bit), Long oder Double-Word  (32 Bit) bzw.  Quad-Word (64-Bit). Für Intels
Syntax leitet der Assembler den Typ automatisch vom Parameter ab.

\hspace{5mm} \makebox[1.5cm]{AT\&T: \hfill}
\texttt{movl \$5, \%eax}

Ein Parameter kann eine absolute Adresse enthalten oder indirekt auf den Inhalt
einer Speicherstelle zeigen. Die Speicheradresse für die Dereferenzierung kann
aus einem Ausdruck berechnet werden.  Diese effektive Adresse wird bei Intel
aus Variablen in eckigen Klammern gebildet und kann eine Typangabe wie
\texttt{BYTE},  \texttt{WORD} (2 Byte) und \texttt{DWORD} (4 Byte) gefolgt von  \texttt{PTR}
(Pointer) vorangestellt bekommen.

\hspace{5mm} \makebox[1.5cm]{Intel: \hfill}
\texttt{mov eax, dword ptr [ebp+4]}

In AT\&T Syntax berechnet sich eine Adresse nach dem sogenannten \emph{base
indexed addressing}-Schema aus den Einzelkomponenten Verschiebung (auch
\emph{displacement}), Basis, Index und Skalierung, wie folgt: $Verschiebung + Basis +
Index*Skalierung$ . Ein Beispiel:

\hspace{5mm} \makebox[1.5cm]{AT\&T: \hfill}
\texttt{movl 4(\%ebp), \%eax}

Die wesentlichen Syntax-Unterschiede sind in der Tabelle~\ref{tab:syntaxdiffs}
zusammengefasst.

%\clearpage

\begin{table}[h]	% place here
\begin{tabular}{lll}
\\	                          & INTEL SYNTAX                  & AT\&T SYNTAX
\\\hline
\\	Parameter 								& \tt mnem dest, src, const  	  & \tt mnem src, dest, const
\\  Adressierung  				  	&	\tt [base+index*scale+disp]   & \tt disp(base, index, scale)
\\	Register      						& \tt eax              					& \tt \%eax
\\	Konstante     						& \tt 0xFF             					& \tt \$0xFF
\\	Dereferenzierung   				& \tt [addr]           					& \tt addr(,1)
\\	Absolute Adresse 			 	  & \tt addr             					& \tt *addr
\\	{\tt byte} Instruktion    & \tt mov byte ptr     					& \tt movb
\\	{\tt word} Instruktion    & \tt mov word ptr     					& \tt movw
\\  {\tt dword} Instruktion   & \tt mov dword ptr    					& \tt movl
\end{tabular}
\caption{Syntax Unterschiede} \label{tab:syntaxdiffs}
\end{table}

\subsection{Assembler-Direktiven}
Jeder Assembler besitzt einen eigenen Satz an spezifischen Direktiven. Streng
genommen gehören diese nicht zur Syntax, allerdings sind sie nötig um einige
Meta-Informationen zur Code-Struktur zu definieren. Die Direktiven sind stark
abhängig von der eingesetzten Software (NASM, YASM, GNU as, etc.) und selten
herrscht Kompatibilität innerhalb der Assembler-Dialekte. Fast immer gibt es aber eine
ähnlich lautende Entsprechung. 

Eine Typische Anweisung ist die Zuordnung eines Bezeichners zu einem Code- oder
Datensegment. Die Adressen der Instruktionen werden beim Assemblieren ihrem
Segment zugeordnet. Bei NASM heißt die Direktive \texttt{section}
bei GAS \texttt{.section}. Die Sektionen \texttt{.text} und
\texttt{.data} für das Code- bzw. statische Datensegment sind dabei traditionell
unter Unix-artigen Systemen gebräuchlich. Die Bezeichner können aber 
beliebig gewählt werden.

Oft ist es auch nötig Datenbereiche zu reservieren oder statisch zu
initialisieren, zum Beispiel um feste Strings zu definieren. Dazu dienen
Pseudo-Instruktionen wie \texttt{DB, DW, DD}, etc. (\emph{define byte, define
word, define dword}) und \texttt{RESB, RESW, RESQ} (\emph{reserve byte}, usw.)
bei NASM und auch MASM. Die Definition von Konstanten erfolgt mittels
\texttt{EQU}. Diesen Definitionen kann ein Label vorangestellt werden über
welches man die Adresse der jeweiligen Daten im Code referenzieren kann. Beim
Assemblieren werden diese Labels in konkrete Adressen übersetzt.

