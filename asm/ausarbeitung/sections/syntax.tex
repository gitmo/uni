\section{Syntax}
In der x86 Assembler-Sprache gibt es zwei populäre Syntaxarten.
Zum einen die von Intel selbst und zum anderen die AT\&T-Syntax.
Erstere wird vor allem in Windows-Umgebungen benutzt, während letztere in der Unix-Welt populär ist.
Die grundlegende Syntax lässt sich auf die sogenannten \textit{instruction operands} reduzieren.

\subsection{Intel Syntax}
\texttt{label: mnemonic [argument1 [, argument2 [, argument3]]]}

Ein mneomic ist lauf Intel's developer Manual folgendermaßen definiert:
,,A mnemonic is a reserved name for a class of instruction opcodes which have
the same function.''
Es kann also als eine generische Funktion auf null bis drei Argumenten verstanden werden.

\subsection{AT\&T Syntax}

\subsection{Segmente}
Jeder x86 Quellcode ist in verschiedene Segmente unterteilt. Da wären zum einen die Datensegmente die mittels den Schüsselwörtern \texttt{.data} und \texttt{.stack} definiert werden, sowie letztendlich das Codesegment. Im letzteren steht das eigentliche Programm, während im ersterem unter anderem Speicher alloziert und initialsiert wird.
