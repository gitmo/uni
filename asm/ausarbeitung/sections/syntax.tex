\section{Syntax}

In der Welt des x86-Assmblers gibt es zwei große Syntaxfamilien: die Intel und
die AT\&T-Syntax. Während Erstere in einer Vielzahl von Assemblern (NASM, TASM,
MASM, YASM, usw.) und Disassemblern (IDA Pro, OllyDbg) zur Standardsyntax
zählt und Intel die x86-Platform damit dokumentiert, ist Letztere dennoch nicht
gänzlich irrelevant. In der UNIX und Linux-Welt wurde lange Zeit nur AT\&T's
Syntaxstil von den ausgelieferten Assemblern \texttt{as} bzw. \texttt{gas}
unterstütlich. Insbesondere durch die Verbreitung der GNU Compiler Collection
hat sich dieser Stil bis heute gehalten, obwohl der GNU Assembler mittlerweile
auch die Intel-Syntax beherscht.

Überhaupt sind die Unterschiede breider Stile gleichmächtig und lassen sich
trotz erheblicher optischer Unterschiede problemlos ineinander konvertieren
(z.B. mit Intel2GAS [i2g]). Für die Folgenden Beispiele wird die Intel-Syntax
als die kanonische Referenz betrachtet und jeweils der äquivilenten
AT\&T-Syntax gegenübergestellt.

% [i2g] http://www.niksula.hut.fi/~mtiihone/intel2gas/

\subsection{Arebitsweise des Assemblers}

Programmiert man mit einem Assembler, wie z.B. NASM, so beschreibt man Zeile
für Zeile je eine Prozessorinstruktionen über ihr Symbol, einem sogenannten
Mnemonic's gefolgt von den dazugehörigen optionalen Argumenten. Die Funktionen
des Assemblers ist es nun dieses Symbolische Programm direkt in die
entsprechende Folge binären x86-Maschinencode – man spricht von Opcodes, die
üblicherweise byteweise, hexadezimal dargestellt werden.

Vereinfach gesagt bedient sich der Assembler einer großen Symboltabelle in der
für jedes Tupel von Mnemnonic und Argumenten der entsprechende Opcode vermerkt
ist. Der Programmierer könnte zwar selbst direkt in Opcodes programmieren und
bekommt aber dadurch die Erleichterung sich statt Hexadezimal-Codes nur Symbole
merken zu müssen.

\subsection{Instruktionen}

Eine Zeile in einem Assemblerprogramm hat folgende Form und ist in beiden
Syntaxstilen gleich:

\texttt{[label:] mnemonic [argument1][, argument2][, argument3]}

Label ist eine Sprungmarke und optional, ebenso wie die Argumente. Deren
erforderliche Anzahl hängt von der Instruktion ab. Maximal drei, häufiger zwei
Argumente sind üblich. Das Mnemnonic bzw. Instruktionssymbol ist eine kurze
Zeichenkette, wie zum Beispiel MOV, ADD oder PUSH, und abstrahiert gleich über
eine Klasse von Opcodes mit derselben Funktion. So ist zum Beispiel die
Kopieroperation für Argumente verschiedene Art (Register, Konstanten) immer das
Mnemnonic MOV obwohl sich die Opcodes unterscheiden.

Groß- und Kleinschreibung wird nicht unterschieden.

\subsection{Parameter}

Bei vielen Befehlen sind zwei Argumente üblich. Es wird dann meist von Quell-
und Zieloperanden gesprochen (source and destination parameter). Die Semantik der
Parameterreihenfolge bei Intel-Syntax entgegengesetzt zur AT\&-Syntax.

Ein Parameter kann von dreierlei Art sein: ein Register, eine Konstante oder
Speicheradresse. Die Registerbezeichnungen (z.B. EAX, EBP) sind bei Intel und
AT\&T identisch, allerdings erfodert Letztere noch ein Prozentzeichen im
Prefix. Konstanten erhalten bei AT\&T ein Dollarzeichen-Prefix. Intel-Assembler
benötigt diese Kennzeichnung nicht.

Eine weitere Eigenart der AT\&T-Syntax ein Befehls-Suffix b, w, l, oder q, der die
Parametergröße beschreibt: Byte, Word (16 Bit), Long oder Double-Word  (32 Bit) bzw.
Quad-Word (64-Bit). Für Intels Syntax leitet der Assembler den Typ automatisch
vom Parameter ab:

\begin{tabular}{ll}
Intel: & \texttt{mov eax, 5} \\
AT\&T: & \texttt{mov \$5,\%eax}.
\end{tabular}

Ein Parameter kann eine absolute Adresse enthalten oder indirekt auf den Inhalt einer Speicherstelle zeigen. Die Speicheradresse für die Derefeerenzierung kann aus einem Ausdruck berechnet werden.
Diese effektive Adresse wird bei Intel aus Variblen in eckigen Klammern gebildet und kann eine
Typangabe wie \texttt{BYTE},  \texttt{WORD} und \texttt{DWORD} gefolgt von  \texttt{PTR} (Pointer) vorangestellt bekommen:

\texttt{mov eax, dword ptr [ebp+4]}

In AT\&T Syntax berechnet sich eine Adresse nach dem sogenannten \emph{base indexed addressing}-Schema aus den Einzelkomponenten
Adresse (auch \emph{disposition}), Basis, Index, Skalierung, wie folgt: \texttt{Adresse + Basis + Index*Skalierung}. Die Entsprechung für obiges Beipiel wäre:

\texttt{MOVL 4(\%EBP), \%EAX}

\begin{tabular}{lll}
\\	   & INTEL SYNTAX                &        AT\&T SYNTAX
\\\hline
\\	Parameter 								& \tt mnem dest, src, const  	  & \tt mnem src, dest, const
\\Adressierung  				   	&	\tt [base+index*scale+disp]  & \tt disp(base, index, scale)
\\	Register      						& \tt eax              					& \tt \%eax
\\	Konstante     						& \tt 0xFF             					& \tt \$0xFF
\\	Dereferenzieung   				& \tt [addr]           					& \tt addr(,1)
\\	Absolute Adresse 				& \tt addr             					& \tt *addr
\\	{\tt byte} Instruktion   & \tt mov byte ptr     					& \tt movb
\\	{\tt word} Instruktion   & \tt mov word ptr     					& \tt movw
\\{\tt dword} Instruktion  & \tt mov dword ptr    					& \tt movl
\end{tabular}



\subsection{Segmente} Jeder x86 Quellcode ist in verschiedene Segmente
unterteilt. Da wären zum einen die Datensegmente die mittels den
Schüsselwörtern \texttt{.data} und \texttt{.stack} definiert werden, sowie
letztendlich das Codesegment. Im letzteren steht das eigentliche Programm,
während im ersterem unter anderem Speicher alloziert und initialsiert wird.
