\section{Quellen}

\begin{thebibliography}{9}

\bibitem{pcworld}
  \href{http://www.pcworld.com/article/146957/birth_of_a_standard_the_intel_8086_microprocessor.html}{
  \emph{Birth of a Standard: The Intel 8086 Microprocessor}}
	von Benj Edwards
	in PC World (17. Juni 2008)
	

\bibitem{postrisc}
	\href{http://www.cse.msu.edu/~enbody/postrisc/postrisc2.htm}{\emph{Beyond RISC - The Post-RISC Architecture}}
	von Mark Brehob, Travis Doom, Richard Enbody, William H. Moore, Sherry Q. Moore, Ron Sass, Charles Severance
	in IEEE Micro Mai 1996
% (IEEE Micro 3/96)

\bibitem{detmerx86}
	\emph{Introduction to 80x86 Assembly Language and Computer Architecture}
	von Richard C. Detmer,
	2001

\bibitem{intelmanual}
	\href{http://www.intel.com/Assets/PDF/manual/253665.pdf}{
	\emph{Intel 64 and IA-32 Architectures Software Developer's Manual - Architecture}}
	von Intel Corporation,
	2011

\bibitem{intelreferenz}
	\href{http://www.intel.com/Assets/PDF/manual/253666.pdf}{
	\emph{Intel 64 and IA-32 Architectures Software Developer's Manual Volume 2A: Instruction Set Reference, A-M}} und
	\href{http://www.intel.com/Assets/PDF/manual/253667.pdf}{
	\emph{Instruction Set Reference, 2B: N-Z}}	von Intel Corporation,
	2011
% \url{http://www.intel.com/products/processor/manuals/}

\bibitem{wp:callconv}
	\href{https://secure.wikimedia.org/wikipedia/en/w/index.php?title=X86_calling_conventions&oldid=414385219}{\emph{x86 calling conventions}} aus englischsprachiger Wikipedia (17. Februar 2011)

\bibitem{wp:abi}
	\href{https://secure.wikimedia.org/wikipedia/en/w/index.php?title=Application_binary_interface&oldid=412119478}{\emph{Application binary interface}} aus englischsprachiger Wikipedia (5. Februar 2011)

\bibitem{i2g}
	\href{http://www.niksula.hut.fi/~mtiihone/intel2gas/}{Intel2GAS}, Software zur Konvertierung zwischen den Syntaxstilen AT\&T und Intel, letzte Version 14.11.2000, aufgerufen am 19.02.2011

\bibitem{Warden1992} Software Reuse and Reverse Engineering in Practice, Warden, R.,1992, London, England: Chapman \& Hall. S. 283–305.

\end{thebibliography}
