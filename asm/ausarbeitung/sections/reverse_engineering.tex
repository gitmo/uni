\section{Reverse Engineering}

Assembler als Haupt-Programmiersprache für Softwareprojekte hat auf Grund der großen Verbreitung einer Vielzahl von Hochsprachen, mit samt deren Vorteilen wie höheres Abstraktionsnieveau, umfangreichere Werkzeuge, Portierbarkeit und Wirtschaftlichkeit von Wartung und Kollaboration in Teams, stark an Bedeutung verloren. Auch sind die Stärken von Assembler wie Schnelligkeit und geringe Codegröße durch das Aufkommen hochoptimierender Compiler und exponentiellem Wachstum von Prozessorleistung und Speicherkapazität in den Hintergrund getreten.

Für ein wichtige Anwendung ist jedoch die Kenntnis und Arbeit mit Assemblercode unverzichtbar: dem sogenannten Reverse Engineering von Software durch Disassemblierung.