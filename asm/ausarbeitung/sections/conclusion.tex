\section{Fazit}
Aus heutiger Sicht hat sich die Verwendung vom x86-Assembler stark verändert.
Wurden vor 20 Jahren noch viele Programme hardwarenah programmiert, so ist dies heutzutage nur noch bedingt der Fall.
Unter Aspekten von Lesbarkeit und Wartbarkeit kann sie nicht mit modernen Hochsprachen wie Java oder Python mithalten.
Dennoch ist ein Grundverständnis von Assembler wichtig, da letztendlich jede Hochsprache in Maschinencode übersetzt wird (der sich zu Assembler eineindeutig zuordnen lässt). 

Ein stark an Bedeutung gewinnender Aspekt in der Informatik ist die Analyse von Schadsoftware. Das Verhalten von Viren und Würmer lassen sich genau mittels \textit{Reverse Engineering} analysieren. Hier ist das Wissen über die vorherrschende Computerarchitekturen im privaten wie auch im unternehmerischen Bereich unabdingbar.

Ein Grundverständnis von Assembler ist eine Bereicherung und befähigt tiefere Einblicke in das Funktionieren von Computern, Elektronik und der Informatik in ihrer Gesamtheit. Einige Grundprobleme lassen sich nicht optimal durch Hochsprachen weg-abstrahieren und so sind hardwarenahe Programmierkenntnisse oft von Vorteil. Dennoch ist x86-Assembler keine \textit{general-purpose Language} für größere Software-Projekte.

Der Assembler kann letztendlich auch als das Latein der Programmiersprachen verstanden werden: Einer Art toten Sprache, die anstatt noch zur verbalen, neu-erschaffenden Kommunikation eingesetzt zu werden, vor allem verstanden, gelesen und übersetzt werden will.