\section{Fazit}
Aus heutiger Sicht hat sich die Bedeutung und Verwendung vom x86-Assembler stark verändert.
Wurden vor 20 Jahren noch viele Programme hardwarenah programmiert, so ist dies heutzutage nur noch bedingt der Fall.
Die Lesbarkeit und Wartbarkeit können nicht mit denen von aktuellen Hochsprachen wie Java oder Python mithalten.
Dennoch ist ein Grundverständnis von Assembler wichtig, da letztendlich jede Hochsprache in einen Assembler ähnlichen Maschinencode übersetzt wird.

Während die Assemblerprogrammierung in der klassischen Softwareentwicklung stark an Bedeutung verloren hat, passiert gegenteiliges gerade bei dem sogenannten \textit{Reverse Engineering}.
Wichtig ist dies unter anderem vor allem bei der Schadsoftware-Analyse (Viren, Würmer und Malware).

Abschließend lässt sich sagen, dass ein Grundverständnis von Assembler durchaus wichtig ist.
Es kann als das Latein des Computers verstanden werden.
Weder wird Latein noch von vielen Menschem im alltäglichen Leben gesprochen, noch wird x86-Assembler bei größeren Software-Projekten als \textit{general-purpose Language} eingesetzt.
Dennoch lassen sich gewisse Grundproblematiken nicht durch Hochsprachen wegabstrahieren und so sind hardwarenahe Programmierkentnisse nicht zu unterschätzen.
