\section{Fazit}
Aus heutiger Sicht ist die Bedeutung von x86-Assembler stark zurückgegangen. Wurden vor 20 Jahren noch ein viele Programme hardwarenah programmiert, so hat sich dies mittlerweile stark verändert. Die Lesbarkeit und Wartbarkeit können nicht mit denen von aktuellen Hochsprachen wie Java oder Python mithalten. Dennoch ist ein Grundverständnis vom x86-Assembler wichtig, da letztendlich jede Hochsprache in Maschinencode übersetzt wird.
Große Bedeutung hat Assembler aber mittlerweile beim sogenannten Reverse Engineering erlangt....
Abschließend lässt sich sagen, dass Assembler das Latein des Computers ist. Lesen und Verstehen reichen für den Größteil der Probleme völlig aus.
