\section{Architektur}
x86-Assembler umfasst eine Instruktionsmenge die ursprünglich für Intels 8086 CPU konzipiert war. Diese basiert auf einer "von Neumann"-Architektur. Hier wird, im Gegensatz zur Harvard-Architektur, für Daten und Programme der gleiche Speicher verwendet. Es existierte ein fester Registersatz von 16 Registern der sich im Laufe der Zeit immer weiter vergrößert hat. Ursprünglich hatte x86 vier general-purpose Registern (ax,bx,cx und dx), vier Segmentierungsregistern zur Speicheradressierung (cs, ds, es und ss), sowie FLAG-Registern und dem Instruktionszeiger-Register.
