%Register
%Adressierung
\section{Architektur}
X86-Assembler umfasst eine Instruktionsmenge die ursprünglich für Intels 8086 CPU konzipiert war. Diese basiert auf einer "von Neumann"-Architektur. Hier wird, im Gegensatz zur Harvard-Architektur, für Daten und Programme der gleiche Speicher verwendet. Es existierte ein fester Registersatz von 16 Registern der sich im Laufe der Zeit immer weiter vergrößert hat. Ursprünglich hatte x86 vier general-purpose Register (AX, BX, CX und DX), vier Segmentregister zur Speicheradressierung (CS, DS, ES und SS), vier Indexregistern (SP, BP, SI, DI), sowie dem FLAG-Register und dem Instruktionszeiger-Register.

\subsection{Prozessor-Register}

Der Registersatz wurde im Laufe der Zeit mit jeder neuen Prozessorgenerationen erweitert. Wichtigster Schritt war dabei die Einführung von 32-Bit Registern mit dem 80386. Alle 16-bit Grundregister mit Ausnahme der Segmentregister wurden in ihrer Größe verdoppelt und mit dem Präfix E versehen. Mit der Einführung von 32-Bit Registern, Adressbus und Instruktionen spricht man auch von der IA-32 Plattform (\emph{Intel Architecture, 32-bit)}.

\subsubsection{Allzweckregister}

\begin{tabular}{|c|l|l|}
\hline EAX & \emph{accumulator} & AX, AH, AL \\
\hline EBX & \emph{base register} & BX, BH, BL \\
\hline ECX & \emph{counter register} & BX, CH, CL \\
\hline EDX & \emph{data register} & DX, DH, DL \\
\hline \end{tabular}

Die vier Allzweckregister, \emph{general-purpose} oder auch Rechenregister dienen als Operanden für vielerlei Instruktionen und ermöglichen eine freie Manipulation der Daten mit denen der Prozessor rechnet.

Sie besitzen seit dem 80386 eine Größe von 32-bit – in der Intel-Sprache ein DWord (\emph{double-word}, wird so bezeichnet da ein Prozessorwort ursprünglich aus 16-bit beinhalten konnte). Die alten Bezeichner AX, BX, CX und DX können nach wie vor verwendet werden und meinen dann die unteren 16-Bit des Registers.

Darüber hinaus war es beim 8086 möglich die Allzweckregister in zwei Teilen anzusprechen: dem \emph{High-Byte} für Bits 8-15 in {\tt AH, BH, CH} für  und {\tt DH} und dem \emph{Low-Byte} mit Bits 0-7 in {\tt AL, BL, CL} und {\tt DL}. Eine Operation auf einen dieser 8-bit Hälften geschieht ohne Seiteneffekte auf jeweils andere.

Obwohl fast alle Rechenoperationen auf alle vier Register angewendet werden können, erfüllen sie für manche Instruktionen spezielle Aufgaben. Genaue Angaben findet der Programmierer dazu zum Beispiel in Intels Referenz-Dokumentation. \cite{intelreferenz}

Ein Compiler legt über seine Aufrufkonvention oft fest ob den Registern besondere Aufgaben zukommen sollen. So wird {\tt EAX} von vielen C-Compilern immer als Register für den Rückgabewert eines Funktionsaufrufs benutzt. \cite{wp:callconv}

\subsubsection{Segmentregister und Adressierung}

\begin{tabular}{|c|c|}
\hline CS & Codesegment \\
\hline DS & Datensegment \\
\hline SS & Stacksegment \\
\hline ES & Extrasegment\\
\hline
\end{tabular}

Die Segmentregister sind als einzige Register 16-Bit breit geblieben und lassen sich nicht in byteweise aufteilen. Das hängt mit ihrer speziellen Funktion zusammen, dann sie nehmen Anfangsadressen von Segmenten im Speicher auf. Die Adresse des Stack-Speichers wird so zum Beispiel über das Registerpaar SS:ESP ermittelt. Vor dem 80386 Prozessor waren sie besonders relevant für die sogenannte segmentierte Adressierung im \emph{Real Mode} um über statt 16-Bit zur Adressierung über 20-Bit zu verfügen.

% lassen sich lediglich auslesen und beschreiben

\subsubsection{Zeige- und Indexregister}

\begin{tabular}{|c|l|}
\hline ESI & \emph{source index} \\
\hline EDI & \emph{destination index} \\
\hline
\end{tabular}

Nützlich als Zeiger auf Datenstrukturen

z.B. String-Manipulation


\begin{tabular}{|c|l|}
\hline ESP & \emph{stack pointer}\\
\hline EBP & \emph{base pointer} \\
\hline
\end{tabular}

Zeiger für den Stack

ESP und EBP dienen zum Aufspannen eines Stackframes bei Funktionsaufrufen

ESP zeigt auf den aktuellen \emph{top}

EBP zeigt auf den \emph{bottom}

\begin{tabular}{|c|l|}
\hline EIP & \emph{instruction pointer} \\
\hline
\end{tabular}

Zeiger auf den nächsten Befehl in der Ausführung

wird ausschließlich intern durch den Prozessor verändert.

\subsubsection{Statusregister}

Ein 16 Bit Flag-Register FLAGS

32 Bit EFLAGS erhält einige weitere Flags

{\small 15}
\begin{tabular}{|c|c|c|c|c|c|c|c|c|c|c|c|c|c|c|c|c|}
\hline & & & & O & D & I & T & S & Z & & A & & P & & C \\
\hline
\end{tabular}
{\small 0}

\emph{overflow, direction, interrupt, trap, sign, zero, auxiliary, parity}
und \emph{carry flag} 

nehmen bestimmte Belegung nach verschiedenen Operationen ein

beeinflussen bestimmte Entscheidungen wie z.B. bedingte Sprünge (\emph{zero flag})


\subsection{Adressraum}

\subsubsection{Real Mode}

16 bit Mode, seit 8086

Segmentregister (CS, DS, SS, ES) sind relevant für segmentierte Adressierung im
\emph{Real Mode} um über statt 16 Bit zur Adressierung, 20 Bit zu verfügen

$Adresse = 16 * Segment + Offset$

Dies erhöht den Adressraum von 64 KByte auf 1 MByte.


\subsubsection{Protected Mode}
Im \emph{protected mode} mit 32-Bit Adressierung werden alle 16-Bit auf 32-Bit erweitert

\texttt{SEGMENT:OFFSET} Adressierung ist ebenfalls möglich, aber weniger relevant als im
\emph{Real Mode}, da nun volle $2^{32}$-Bit (4 GByte) adressiert werden können




Arten der Adressierung:

\begin{enumerate}
\item Direktwertadressierung (Immediates)
\item Direkte Adressierung
\item Indirekte Registeradressierung
\item Indizierte Registeradressierung
\end{enumerate}


% \section{Unterprogramme}
