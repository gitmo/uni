\section{Hello World}

In diesem Abschnitt soll ein einführendes Beispiel zur Programmieren in x86-Assembler gegeben werden. Um in der Tradition von Programmiersprachen-Literatur zu bleiben, wird hier ein kurzes Programm demonstriert, das ``Hello, World!'' auf der Konsole ausgibt.

Assembler-Programme sind naturgemäß stark Plattformabhängig, da auf niedrigster Ebene mit dem Betriebssystem kommuniziert wird.
Das folgende Listing ist für \emph{Mac OS X} entwickelt worden, und
funktioniert auch auf den nah verwandten BSD-Unix-Systemen. Mit kleinen Änderungen kann es für Linux angepasst werden, die in einem extra Abschnitt beschrieben werden.

\subsection{Assemblieren und Binden}

Das Programm lässt sich mit folgender Shell Befehlen übersetzen und ausführen.

\begin{lstlisting}[caption=Assemblieren und Binden von hello.asm]
$ nasm -f macho32 hello.asm
$ ld hello.o -o hello
$ ./hello
Hello, World!
\end{lstlisting}

Der Assembler NASM (Netwide Assembler) übersetzt das Listing in Objektcode im Mach-O-Format – dabei handelt es sich um das Dateiformat für ausführbaren Code unter \emph{Mac OS X}. Das Ergebnis ist der Zwischencode in \texttt{hello.o}. Im zweiten Schritt wird der Linker \texttt{ld} angewiesen eine ausführbare Programmdatei (\emph{ Executable}) mit dem Namen \texttt{hello} zu erzeugen, welche im Anschluss ausgeführt wird. Die Vorgehensweise beim Binden ist analog zur Arbeit mit dem C-Compiler von \emph{GCC} der ebenfalls im Zwischenschritt Objektcode-Dateien aus C-Programmen erzeugt.

\subsection{Code für BSD-Unix Systeme}

Im Folgenden soll das Programm genauer erläutert werden. Genaugenommen handelt es sich nur bei den Zeilen ab dem Label \texttt{kernel} um Assembler-Instruktionen. Die vorhergehenden Zeilen sind Präprozessor-Makros (Prefix \%) und Assembler-Direktiven. Der Linker benutzt das Label {\tt start} als Einsprungspunkt in das \emph{Executable} (Linux: {\tt \_start}).
Dieses Label wird mit der Direktive {\tt global} als nach außen sichtbar markiert.

\begin{lstlisting}[numbers=left,caption=hello.asm]
; Preprocessor macros
%define stdout  1

%define SYS_exit    1       ; System calls
%define SYS_write   4

; segment for static data
section .data

message   db 'Hello, World!', 0Ah  ; End with newline
length    equ $ - message          ; Length = current - previous address

; code segment
section .text
global start                ; Make start function externally visible for linker

kernel:                     ; Expects syscall number in eax register
    int     80h             ; Call kernel
    ret

start:                      ; entry symbol on Mac OS X

    push    length          ; size_t nbyte
    push    message         ; const void *buf
    push    stdout          ; int filedescriptor
    mov     eax, SYS_write  ; Make the system call to write
    call    kernel

    add     esp, 12         ; 3 args * 4 bytes each = 12 bytes

    push    0               ; exit status returned to the operating system
    mov     eax, SYS_exit   ; Make the syscall to exit
    call    kernel
\end{lstlisting}

Das Listing besteht aus dem Hauptprogramm, gekennzeichnet mit dem Label \texttt{start} und einem Unterprogramm mit dem Label \texttt{kernel}. Zur Ausgabe der Textnachricht wird hier direkt auf den System-Call \texttt{write} des Betriebssystemkernels zurückgegriffen.

Das Hauptprogramm bereitet die Parameter für {\tt write} vor, indem es sie nach C-Aufruf"-konvention von hinten nach vorne mittels {\tt push} auf dem Stack ablegt. Der Stackpointer {\tt esp} wird mit jedem Push um 4 Bytes ($=$ 32 Bit) verringert -- zur Erinnerung: der Stack wächst von den hohen zu niedrigen Speicheradressen. Welche Parameter der System-Call benötigt beschreibt die Manual-Page {\tt man 2 write}:

\begin{lstlisting}
    ssize_t write(int fildes, const void *buf, size_t nbyte);
\end{lstlisting}

\begin{enumerate}
\item {\tt length}: Länge der Nachricht – eine Konstante definiert durch die
NASM Anweisung {\tt equ} als Abstand in Bytes Zwischen dem Label length und message.
\item {\tt message}: Die Adresse der Nachricht definiert durch {\tt db} als String von ASCII-Bytes.
\item {\tt stdout}: Der Dateideskriptor der Standardausgabe (=1) unter Unix-artigen Systemen.
\end{enumerate}

Der Kernel wird mit Hilfe des Software-Interrupts 80h (hexadezimal 80, identisch unter Linux, BSD und OS X) angesprungen. Einen ganzzahliger Wert im Register \texttt{eax} dient als Angabe welcher System-call verwendet werden soll. Für {\tt write} ist das die Konstante 4. Nach Behandlung des Interrupts wird mittels \texttt{ret} an die nächste Instruktion im Hauptprogramm zurückgekehrt an der vorher mittels \texttt{call kernel} der Prozeduraufruf stattfand.

Der Stack enthält nun immer noch die drei Parameter an der Stelle des Stackpointers ({\tt esp}), die durch das Addieren von 12 Bytes -- jeder Parameter belegte 4 Bytes -- freigegeben werden können. Da das Programm aber als nächstes mit dem Syscall \texttt{exit} beendet wird, ist dies optional, da der Stack dann auch vom Betriebssystem freigegeben wird.


\subsection{Anpassungen für Linux}

Wie bereits erwähnt wurde, sind beim Aufruf von Betriebssystemfunktionen feste Konventionen \cite{wp:callconv} der Parameterübergabe einzuhalten. Diese unterscheiden sich von System zu System (definiert in der ABI \cite{wp:abi}) und sogar von einem Compiler zum Anderen.

Der Linux-Kernel erwartet die sogenannten \emph{fastcall}-Konvention, wobei die Parameter über festgelegte Register übergeben werden, insofern es deren Typ und Anzahl erlaubt. Dies ist oft effizienter, da es die Anzahl der teueren Speicherzugriffe auf den Stack minimiert. Prozessorregister arbeiten erheblich schneller als Cache oder Hauptspeicher. Das erste Argument wird dazu in {\tt ebx}, das zweite in {\tt ecx}, usw. übergeben.

\begin{lstlisting}[caption=Linux-Kernel Aufrufkonvention]
    ; Linux: fastcall convention
    mov     edx, length
    mov     ecx, message
    mov     ebx, stdout
    mov     eax, SYS_write  ; Make the system call to write
    call    kernel
\end{lstlisting}

Die zweite kleine Änderung betrifft das Label {\tt start}. Der Linker sucht unter Linux den Einsprungspunkt stattdessen bei {\tt \_start}.
