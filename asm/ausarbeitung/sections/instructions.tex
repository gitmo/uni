\section{Instruktionen}

\subsection{Daten kopieren}
%Introduction to 80x86 Assembly Language and Computer Architecture - Seite 86
Einer der am häufigsten verwendeten Operationen ist das Kopieren von Daten von einer Stelle zu einer anderen.
Hierzu wird im x86-Assembler das \texttt{mov}-Mnemonic verwendet. Es hat die folgende Syntax:

\begin{verbatim}mov    Zieladresse, Quelle\end{verbatim}

Es können immer nur Daten der gleichen Größe kopiert werden.
Das heißt es muss sichergestellt sein, dass die Register beziehungsweise Adressen dementsprechend gleich große Werte speichern können.
Im folgendem Beispiel wird in das Register eax der Wert aus dem Register ebx kopiert:

\begin{verbatim}mov    eax, ebx\end{verbatim}

Um einer Zieladresse einen festen Wert zu zuweisen kann statt einer Adresse im zweiten Parameter auch eine Konstante stehen:

\begin{verbatim}mov    eax, 5\end{verbatim}

Hier wird dem Register \textit{eax} der Wert 5 zugewiesen.

Wichtig ist, dass es sich bei einer der beiden Adressen um ein Register handeln muss.


\subsection{einfache mathematische Operationen}
%inc,dec,add,sub,xor
%Introduction to 80x86 Assembly Language and Computer Architecture - Seite 96
Der x86-Assembler verfügt unter anderem über folgende mathematische Operationen

\begin{description}
	\item [INC] Inkrementiert das Ziel
		\begin{verbatim}INC Ziel\end{verbatim}

	\item [ADD] Addiert Ziel und Quelle und speichert das Ergebnis in Ziel
		\begin{verbatim}INC Ziel, Quelle\end{verbatim}

	\item [DEC] Dekrementierung  das Ziel
		\begin{verbatim}INC Ziel\end{verbatim}

	\item [SUB] Subtrahiert Quelle von Ziel und speichert das Ergebnis in Ziel
		\begin{verbatim}INC Ziel, Quelle\end{verbatim}
\end{description}

In den sogenannten \textit{Flag}-Registern kann überprüft werden ob das Ergebnis negativ oder null ist, sowie festgestellt werden ob es einen Überlauf gab.

\subsection{jmp, je ,jne}

\subsection{cmp}

\subsection{push/pop und call/leave}
