\section{Instruktionen}

Anders als in Hochsprachen wird bei der Assemblerprogrammierung meist mit Registern gearbeitet.
Auf diese können dann Instruktionen angewendet werden. Diese sind in ihrer Anzahl limitiert.

\subsection{Daten kopieren}
%Introduction to 80x86 Assembly Language and Computer Architecture - Seite 86
Einer der am häufigsten verwendeten Operationen ist das Kopieren von Daten vom Speicher in ein Register und wieder zurück.
Darunter fällt auch das Zuweisen eines konstanten Wertes eines Registers oder einer Speicherzelle.
Hierzu wird im x86-Assembler das \texttt{mov}-Mnemonic verwendet.
Es können immer nur Daten der gleichen Größe kopiert werden.
Das heißt es muss sichergestellt sein, dass die Register beziehungsweise Adressen dementsprechend gleich große Werte speichern können.
Wichtig ist, dass es sich bei einer der beiden Adressen immer um ein Register handeln muss.

Im folgendem Beispiel wird in das Register \texttt{eax} der Wert aus dem Register \texttt{ebx} kopiert:

\begin{lstlisting}
mov    eax, ebx
\end{lstlisting}

Um einer Zieladresse einen festen Wert zu zuweisen, kann statt einer \textit{Source-Adresse} auch eine Konstante stehen:

\begin{lstlisting}
mov    eax, 5
\end{lstlisting}

Im Register \textit{eax} steht nun nach erfolgreicher Abarbeitung der Instruktion der Wert 5.


\subsection{Arithmetische und logische Funktionen}
%inc,dec,add,sub,xor
%Introduction to 80x86 Assembly Language and Computer Architecture - Seite 96
Meist werden die mathematische Basisoperationen in der Hardware mit Hilfe der ALU\footnote{Arithmetisch-logische-Einheit} berechnet.
Hierzu zählen arithmetische Funktionen, wie die Addition und Subtraktion, sowie logische Funktionen.

Mittels \texttt{INC} und \texttt{DEC} können Register in- beziehungsweise dekrementiert werden. Um zwei Register zu addieren oder subtrahieren, wird das \texttt{ADD}- beziehungsweise das \texttt{SUB}-Mnemonic verwendet.

Da bei einigen Operationen Sonderfälle auftreten können, werden Informationen diesbezüglich in den sogenannten \textit{Flag}-Registern gespeichert.
Hier kann überprüft werden ob das Ergebnis negativ oder null ist, sowie festgestellt werden ob es einen Überlauf gab.

\paragraph{Beispiel\newline}\makebox{}

\begin{lstlisting}
mov    eax, 255  ; Register eax wird der Wert 255 zugewiesen
add    eax, 5    ; Addition: eax + 5 = 260 
\end{lstlisting}

In unserem Beispiel für 8-Bit große Register führt die Addition zu einem \textit{Overflow}.
Dies ist ein Sonderfall und dadurch wird das entsprechende \textit{O}-Flag\footnote{Overflow Flag} gesetzt.

Ein Blick auf die \textit{Flag}-Register zeigt uns genau dies:

\begin{tabular}{|c|c|c|c|c|c|c|c|c|c|c|c|c|c|c|c|c|c|}
\hline Werte & 0 & 0 & 0 & 1 & 0 & 0 & 0 & 0 & 0 & 0 & 0 & 0 & 0 & 0 & 0 & 0 \\
\hline Register & & & & O & D & I & T & S & Z & & A & & P & & C & \\
\hline
\end{tabular}

\subsection{Vergleiche}
Bei der Lösung von Problemen ist man als Programmierer oft auf Vergleiche zwischen Werten angewiesen.
Hier gibt zwei Mnemonics um Operanten zu vergleichen. Zum einen das \texttt{test}- und das \texttt{cmp}-Mnemonic.
Der Unterschied zwischen beiden ist die mathematische Umsetzung der Operation. Während beim ersteren beide Operanten logisch miteinander verundet werden, werden beim letzteren diese einfach voneinander abgezogen. 
Das Ergebnis kann nur anhand der gesetzten \textit{Flags} ausgelesen werden.

\paragraph{Überblick}
\begin{itemize}
	\item \texttt{cmp arg1, arg2} $\leftrightarrow \text{arg1} - \text{arg2}$

	\item \texttt{test arg1, arg2} $\leftrightarrow \text{arg1} \wedge \text{arg2}$
\end{itemize}

\paragraph{Beispiel\newline}
\makebox{}
\begin{lstlisting}
mov    eax, 5
mov    ebx, 5
cmp    eax, ebx
\end{lstlisting}

Anhand des gesetzten \textit{Z}-Flags\footnote{Zero Flag} ist es klar, dass beide Argumente den gleichen Wert haben ($5 = 5$). 

\begin{tabular}{|c|c|c|c|c|c|c|c|c|c|c|c|c|c|c|c|c|c|}
\hline Werte & 0 & 0 & 0 & 0 & 0 & 0 & 0 & 0 & 1 & 0 & 0 & 0 & 0 & 0 & 0 & 0 \\
\hline Register & & & & O & D & I & T & S & Z & & A & & P & & C & \\
\hline
\end{tabular}

\subsection{Verzweigungen und bedingte Anweisungen}

\subsubsection{unbedingter Sprung}
Ein unbedingter Sprung kann hilfreich sein um zum einem einen besseren Überblick zu gewährleisten und zum anderen Wiederverwendbarkeit von Quelltext zu ermöglichen.

Ein unbedingter Sprung wird mittels Manipulation des \textit{Instruction Pointer} realisiert.
Statt die Adresse der sequentiell darauffolgende Instruktion zu laden wird die Adresse der anzuspringenden Instruktion gesetzt.

Um die Adresse zu ermitteln wird hier meist mit sogenannten \textit{Labels} gearbeitet. Diese werden vor der eigentlichen anzuspringenden Instruktion geschrieben. Der Compiler ersetzt dann dieses Label mit der richtigen Adresse. Es können desweiteren auch Register angegeben. In diesem Falle wird der Inhalt des Registers als Adresse interpretiert und angesprungen.

\paragraph{Beispiel\newline}\makebox{}
\begin{lstlisting}
jmp    exit   ; Springe zum Label exit
.
.
.
exit: ...     ; Label exit
\end{lstlisting}

\subsubsection{bedingter Sprung}
Bei einem bedingten Sprung wird je nach erfolgreicher Bedingung der Sprung ausgeführt oder ignoriert. Die Bedingung wird mittels Mnemonic beschrieben. Die für den Vergleich benötigten Werte, werden aus den \textit{Flag}-Registern ausgelesen.

Hierzu wird meist eine Vergleichs-Operation vor dem eigentlichen Sprung benutzt. Diese setzt die \textit{Flag}-Register neu. Je nachdem ob nun die Bedingung anhand dieser Register-Werte wahr oder falsch ist, wird die Sprunganweisung ausgeführt oder eben nicht. 

Beachtet werden sollte, dass eine bedingte Sprunganweisung die \textit{Flag}-Werte nicht verändert. Das heißt, dass der Programmier dafür zuständig ist, dass die richtigen Werte vor dem Vergleich in diesen Registern stehen. Das \texttt{CMP}-Mnemonic zum Beispiel setzt diese anhand des Ergebnises, das \texttt{MOV}-Mnemonic dagegen nicht.

Die Dauer eines bedingten Sprunges beträgt seit Mitte der 90er Jahre\footnote{Einführung des Pentium} einen Taktzyklus.

\paragraph{oft benutzte bedingte Sprünganweisungen}
\begin{description}
	\item [JZ] jump if zero
	\item [JG] jump if greater 
	\item [JE] jump if equal 
	\item [JL] jump if less 
\end{description}

%Seite 151
\paragraph{if Statement}

Da es sich bei dem x86-Assembler nicht um eine Hochsprache handelt, verfügt die Sprache unter anderem nicht über Konstruke wie das \textit{if}-Statement.
Hier kann der bedingte Sprung mit einem vorherigen Vergleich Abhilfe leisten.
Hierzu wird erst die Bedingung überprüft und das Ergebnis in den \textit{Flag}-Registern abgespeichert.
Je nach gesetzten Registerwerten wird nun der Sprungbefehl ausgeführt oder eben nicht.

\paragraph{Pseudocode\newline}\makebox{}
\begin{lstlisting}
if(ecx != edx)
{
	...
}
//ifEnd
\end{lstlisting}

\paragraph{x86 Assembler\newline}\makebox{}
\begin{lstlisting}
        cmp ecx, edx
        je ifEnd ; Sprung wenn ecx == edx
        ...
ifEnd:  ; Sprungadresse wenn ecx und edx
        ; gleich sind 
\end{lstlisting}

Auf gleicher Art und Weise können unter anderem \textit{if/else} und \textit{switch}-Konstrukte realisiert werden.


\subsection{Stackoperationen}
%s 194
\subsubsection{x86 Stack}
Der x86-Stack verhält sich analog zu der bekannten Stack-Datenstruktur. Hierbei handelt es sich um einen sogenannte FIFO-Speicher.
Die Größe des Speichers wird am Anfang eines jeden x86-Programms gesetzt.

\paragraph{Daten auf dem Stack hinterlegen\newline} 
Mit dem \texttt{PUSH}-Mnemonic können Werte auf dem Stack hinterlegt werden.
Hierbei kann es sich um ein Register oder eine Speicherzellen handeln.
Bei Ausführung der Instruktion wird nun der Inhalt des angegebenen Speichers in den Stackspeicher kopiert
und der sogenannte \textit{ESP}\footnote{Register des \textit{Stack-Pointers}} dekrementiert.
Der Stack wächst nämlich nach unten.

%s.196

\textit{Stack-Operation}
\begin{lstlisting}
push    10  ; Speichert den Wert 10 auf dem Stack
\end{lstlisting}

\begin{multicols}{2}
\begin{minipage}{5cm}
\emph{Stack vor der Operation}\\
\begin{tabular}{c|c|}
	\cline{2-2}
   0xFF & 42\\ \cline{2-2}
   0xFE & \\ \cline{2-2}
   0xFD & \\ \cline{2-2}
	      & ... \\ \cline{2-2}
	 0x00 & \\ \cline{2-2}
\end{tabular}
ESP: 0xFF
\end{minipage}

\begin{minipage}{5cm}
\emph{Stack nach der Operation}\\
\begin{tabular}{c|c|}
	\cline{2-2}
   0xFF & 42\\ \cline{2-2}
   0xFE & 10\\ \cline{2-2}
   0xFD & \\ \cline{2-2}
	      & ... \\ \cline{2-2}
	 0x00 & \\ \cline{2-2}
\end{tabular}
ESP: 0xFE
\end{minipage}
\end{multicols}

Leicht zu erkennen ist, dass der \textit{Stack-Pointer} immer auf das zuletzt hinzugefügte Element im Stack zeigt.


\paragraph{Daten vom Stack holen\newline}
%s.196
Um die hinterlegten Werte wieder abzurufen wird das \texttt{POP}-Mnemonic verwendet.
Hierzu wird das oberste Element des Stacks in eine angegeben Ziel-Adresse kopiert. Als Ziel kann ein Register oder eine Speicherzelle fungieren.  

\textit{Stack-Operation}
\begin{lstlisting}
push    eax  ; Speichert das zuletzt hinzugefügte Element
             ; im Stack in das Register eax
\end{lstlisting}

\begin{multicols}{2}

\begin{minipage}{5cm}
\emph{Stack vor der Operation}
\begin{tabular}{c|c|}
	\cline{2-2}
   0xFF & 42\\ \cline{2-2}
   0xFE & 10\\ \cline{2-2}
   0xFD & \\ \cline{2-2}
	      & ... \\ \cline{2-2}
	 0x00 & \\ \cline{2-2}
\end{tabular}
ESP: 0xFE
\end{minipage}

\begin{minipage}{5cm}
\emph{Stack nach der Operation}
\begin{tabular}{c|c|}
	\cline{2-2}
   0xFF & 42\\ \cline{2-2}
   0xFE & 10\\ \cline{2-2}
   0xFD & \\ \cline{2-2}
	      & ... \\ \cline{2-2}
	 0x00 & \\ \cline{2-2}
\end{tabular}
ESP: 0xFF
\end{minipage}
\end{multicols}

Leicht zu erkennen ist, dass die Werte im Stack physikalisch nicht gelöscht werden. Jediglich der ESP  wird verändert.

\subsubsection{Funktionen und Prozeduren}
%call/leave s201
%Intel Software Developers Manuel Volume 1, 6.3.1 Near CALL and RET Operation
%S 179

Um Programme in Unterprogramme unterteilen zu können, sowie Quellcode zu isolieren um Wiederverwendbarkeit zu ermöglichen, sind Funktionen von hohem Nutzen.\cite[S. 179]{intelmanual}

Leider sind Funktionen und Prozeduren nicht in der aus heutiger Sicht bekannten Art und Weise im x86-Assembler implementiert, sondern deutlich rudimentärer.
Mnemonics wie \texttt{call} und \texttt{ret} ermöglichen einem das Aufrufen und Verlassen von Subroutinen.

\paragraph{Eintreten in eine Subroutine}
Das \texttt{call}-Mnemonic fungiert als Aufruf und sichert den aktuellen Zustand der Register.
Es gibt zum einen den sogenannten \textit{near call} und den \textit{far call}.
Der Unterschied liegt darin, ob das bisherige Codesegment neu geladen werden muss oder nicht.
Das \texttt{call}-Mnemonic sichert den \textit{Instruction Pointer} (IP) und wenn nötig auch das Codesegment auf dem Stack ab und lädt diese für die anzuspringende Subroutine in die jeweiligen Register.

Die Ausführung eines \texttt{near calls} lässt sich in folgende Unterschritte unterteilen:

\begin{enumerate}
	\item Abspeichern des aktuellen \textit{Instruction Pointers} auf dem Stack
	\item Laden des neuen \textit{Instruction Pointers} in das Register \texttt{EIP}
	\item Begin der Abarbeitung der Subroutine	
\end{enumerate}

\paragraph{Verlassen einer Subroutine}

Um wieder in das aktuelle Programm zurück zu springen ist es nötig, dass die vorher gesicherten Werte wiederhergestellt werden.
Mithilfe des \texttt{ret}-Mnemonic wird der \textit{Instruction Pointer} wieder geladen und gegebenenfalls das ursprüngliche Codesegment geladen.

Auch das \texttt{ret}-Mnemonic lässt sich für einen \textit{near call} in fogende Teilschritte unterteilen:
\begin{enumerate}
	\item \textit{Instruction Pointer} in das \texttt{EIP} Register laden
	\item Modifizierung des \textit{Stack Pointers} je nach Anzahl der zurückgebenden Parametern. 
	\item Fortführung des Hauptprogramms 	
\end{enumerate}

\paragraph{Beispiel\newline}\makebox{}
\begin{lstlisting}
myFunc: mov    eax, 10 ; Speichert im Register eax den Wert 10
        ret            ; Stellt den IP wiederher
				               ; und springt somit zurück zum Aufrufer

main:   call myFunc    ; Ruft die Subroutine myFunc auf
\end{lstlisting}
