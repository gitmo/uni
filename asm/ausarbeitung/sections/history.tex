\section{Geschichte}

Die Dominanz der x86-Architektur war vor 30 Jahren keineswegs vorauszusehen.
Ende der 70er Jahre war die Firma Intel eine von vielen Chip-Herstellern in der westlichen Welt.
Desweiteren hatte Intel mit Zilog Incorporation, einer Firma ehemaliger Intel Mitarbeiter, einen starken aufstrebenden Konkurrenten.
Dieser machte ihnen mit ihrem 8080-Mikroprozessor-Klon Z80 stark zu schaffen. 

Die Hoffnung mit der neuen Prozessorgeneration in Form des iAPX 432 einen Vorteil gegenüber der starken Konkurrenz zu gewinnen, musste immer wieder verschoben werden.
Der Wechsel auf eine 32-Bit-Architektur schien für die damalige Chipindustrie eine größere Herausforderung zu sein als erwartet und so traten stets neue Probleme auf.

Um den Anschluß nicht ganz zu verlieren, versuchte Intel mit einem kleinen Team um den Software-Entwickler Stephen Morse eine Antwort auf den damals populären Z80 zu finden.
Der von Stephen Morse entwickelte 8086 16Bit-Mikroprozessor war dennoch nur bedingt ein Erfolg.
Ändern sollte sich dies erst, als IBM 1981 den sogenannten preisgünstigen \textit{Personal Computer} entwickelte. Das IBM Model 5150 war ein Kassenschlager und läutete die PC-Revolution in Unternehmen und später auch im privaten Bereich ein.
Verbaut war hier ein 8088-Mikroprozessor von Intel. Dieser war der Nachfolger des 8086-Mikroprozessor und verwendete somit die gleiche Instruktionsmenge.
Hierdurch wurde der Grundstein für den bis heute anhaltenden Siegeszug der x86-Architektur gelegt.\cite{pcworld} 

Ein Grund für die heute Monopolstellung der x86-Architektur ist seine Abwärtskompatibilität.
Jedes für den 8086-Prozessor geschriebene Programm kann noch heute ohne Modifizierungen auf der neusten x86-Prozessorgenerations ausgeführt werden.
Umgesetzt wird dies mit Hilfe sogenannter \textit{Operations modes}. Zu dem anfänglichen \textit{Real mode} des 8086-Prozessors, kamen mit der Zeit immer weitere hinzu.
Hierdurch konnte unter anderem der addressiebare Addressraum erweitert, virtueller und geschützter Speicher eingeführt werden.
Die Word-Größe des x86-Assemblers wurde im Laufe der Zeit zweimal vergrößert. Das erste Mal im Jahre 1985. Hier wurde von Intel der Wechsel von 16 zu 32 Bit vollzogen. In den Jahren 1999 bis 2003 erweiterte AMD die bis dato 32-Bit x86-Architektur auf 64 Bit.

Die x86-Architektur lässt sich ursprüglich der CISC\footnote{\textit{complex instruction set computer}}-Architektur zu ordnen.
Neueste Prozessorgenerationen bringen jedoch diese Eindeutigkeit ins wanken, da hier der Prozessor selbst gewisse Befehle auf eine RISC\footnote{\textit{reduced instruction set computer}}-Architektur  reduziert.
Aufgrunddessen wird in manchen Kreisen mittlerweile auch von der sogenannten \textit{POST-RISC}-Archtiktur gesprochen.\cite{postrisc}
